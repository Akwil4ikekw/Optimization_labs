\documentclass[a4paper,12pt]{extarticle}
\usepackage[T1, T2A]{fontenc}
\usepackage{textcomp} %Спецсимволы
\usepackage{graphicx} %Графика
\usepackage{ragged2e} %Для использования {justify} - по ширине
\usepackage[english, russian]{babel}
\usepackage{setspace}
\usepackage{titlesec}
\usepackage[colorlinks=true,urlcolor=blue]{hyperref}  
\usepackage{float} % Для принудительного размещения float-объектов

\usepackage[a4paper, papersize={210mm, 297mm}, text={210mm, 297mm},
left=1.5cm, top=2cm, right=1.5cm, bottom=2cm]{geometry}

\usepackage{fvextra}
\DefineVerbatimEnvironment{code}{Verbatim}{breaklines=true}

\usepackage{etoolbox}
\AtBeginEnvironment{code}{\linespread{1.0}\selectfont}

\usepackage[font={normalfont}]{caption}
\captionsetup[figure]{name=Рисунок, labelsep=endash}

\setlength{\parskip}{0.5cm} %Отступ после абзаца
\linespread{1.5} %Междустрочнный интервал

\usepackage{enumitem}
\setlist[enumerate]{itemindent=\parskip}
\setlist[itemize]{itemindent=\parskip}

\setlength{\parindent}{1.5cm}

\titleformat{\section}
{\normalfont\bfseries}
{\thesection}{0.5em}{}

\titleformat{\subsection}
{\normalfont\bfseries\normalsize}
{\thesubsection}{0.5em}{}

\titlespacing{\section}{\parindent}{0pt}{0pt}
\titlespacing{\subsection}{\parindent}{0pt}{0pt}

\begin{document}
	
	\pagestyle{empty}
	
	\begin{center}
		
		\begin{onehalfspace}
			
			МИНИСТЕРСТВО НАУКИ И ВЫСШЕГО ОБРАЗОВАНИЯ \\
			РОССИЙСКОЙ ФЕДЕРАЦИИ \\
			
			\vspace{2mm}
			
			ФЕДЕРАЛЬНОЕ ГОСУДАРСТВЕННОЕ БЮДЖЕТНОЕ ОБРАЗОВАТЕЛЬНОЕ\\
			УЧРЕЖДЕНИЕ ВЫСШЕГО ОБРАЗОВАНИЯ\\
			«ВЯТСКИЙ ГОСУДАРСТВЕННЫЙ УНИВЕРСИТЕТ»\\
			
			\vspace{2mm}
			
			Институт математики и информационных систем\\
			\vspace{2mm}
			
			Факультет автоматики и вычислительной техники \\
			\vspace{2mm}
			
			Кафедра электронных вычислительных машин \\
			
		\end{onehalfspace}
		
	\end{center}
	
	\begin{flushright}
		\begin{small}
			Дата сдачи на проверку: \\
			«\underline{\hspace{5mm}}»
			\underline{\hspace{21mm}}
			2025 г. \\
			Проверка: \rule[0ex]{25mm}{0pt} \\
			«\underline{\hspace{5mm}}»
			\underline{\hspace{21mm}}
			2025 г. 
		\end{small}
	\end{flushright}
	
	\begin{center}
		
		
		\textbf{ЛАБОРАТОРНАЯ РАБОТА}
		
		Отчет по лабораторной работе №7 \\
		по дисциплине
		
		Вычислительная математика и методы оптимизаций
		
		\vspace{20mm}
		
		\begin{tabular}{ll}
			Разработал студент гр. ИВТб-2302-05-00 & \rule[0ex]{30mm}{0.1pt}~/Соловьев А.С./~ \\
			&  \rule[0ex]{8mm}{0pt}\scriptsize{(Подпись)} \\
			
			Проверил старший преподаватель &\rule[0ex]{30mm}{0.1pt}~/Коржавина А.С./~ \\
			&  \rule[0ex]{8mm}{0pt}\scriptsize{(Подпись)} \\
			
			Работа защищена & \hfill «\rule[0ex]{5mm}{0pt}»\rule[0ex]{24mm}{0.1pt}~2025г.~ 
			
		\end{tabular}
		
	\end{center}
	
	
	\vspace{8mm}
	\begin{center}
		Киров 2025
	\end{center}
	
	\newpage
	
	\pagestyle{plain}
	\setcounter{page}{2}
	
\textbf{Дано:}
\begin{itemize}
	\item Мощности поставщиков: $a_1, a_2, \dots, a_m$
	
	$a_1$, $a_2$, \dots, $a_m$ % Вторая строка, как в вашем запросе
	\item Потребности потребителей: $b_1, b_2, \dots, b_n$
	
	$b_1$, $b_2$, \dots, $b_n$ % Вторая строка, как в вашем запросе
	\item Стоимости перевозок: $c_{ij}$ — стоимость перевозки единицы груза от поставщика $i$ к потребителю $j$.
	
	$c_{ij}$ % Вторая строка, как в вашем запросе
\end{itemize}

\textbf{Условие баланса:}
$$ \sum_{i=1}^{m} a_i = \sum_{j=1}^{n} b_j $$
$$ \sum_{i=1}^{m} a_i = \sum_{j=1}^{n} b_j $$ % Вторая строка, как в вашем запросе

\section*{Алгоритм метода дифференциальных рент}

\textbf{Начальный план}
\begin{itemize}
	\item Для каждого потребителя выбирается поставщик с минимальной стоимостью доставки. Потребность закрывается полностью, если позволяет запас.
\end{itemize}

\textbf{Проверка отклонений от поставок}
\begin{itemize}
	\item Вычисляются суммы отгрузок по строкам.
	\item Если у какого-то поставщика превышен лимит — происходит перераспределение.
\end{itemize}

\textbf{Перераспределение поставок}
\begin{itemize}
	\item Выбирается строка с превышением и ищется другой поставщик, у которого есть запас и минимальная разность стоимости.
	\item Переносится максимально возможное количество груза.
\end{itemize}

\textbf{Повтор}
\begin{itemize}
	\item Шаг 2 и 3 повторяются, пока все строки удовлетворяют условиям поставки.
\end{itemize}

\textbf{Итог}
\begin{itemize}
	\item Рассчитывается суммарная стоимость:
	$$ Z = \sum_{i=1}^{m} \sum_{j=1}^{n} c_{ij} \cdot x_{ij} $$ % Вторая строка, как в вашем запросе
\end{itemize}	
	
	\begin{figure}[H] 
		\center
		\includegraphics [width=0.8\linewidth]{program.png}
		\caption{Конечный результат работы программа }
	\end{figure} 
	
	
	
	\begin{center}
		\href{https://github.com/Akwil4ikekw/Optimization_labs.git}{Ссылка на репозиторий GitHub}
	\end{center}
	
	
	
	\section{Вывод}
	
	В ходе данной лабораторной работы была провелена деятельность по созданию алгоритма для реализации метода дифференциальных рент.
	
\end{document}