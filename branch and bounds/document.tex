\documentclass[a4paper,12pt]{extarticle}
\usepackage[T1, T2A]{fontenc}
\usepackage{textcomp} %Спецсимволы
\usepackage{graphicx} %Графика
\usepackage{ragged2e} %Для использования {justify} - по ширине
\usepackage[english, russian]{babel}
\usepackage{setspace}
\usepackage{titlesec}
\usepackage[colorlinks=true,urlcolor=blue]{hyperref}  
\usepackage{float} % Для принудительного размещения float-объектов

\usepackage[a4paper, papersize={210mm, 297mm}, text={210mm, 297mm},
left=1.5cm, top=2cm, right=1.5cm, bottom=2cm]{geometry}

\usepackage{fvextra}
\DefineVerbatimEnvironment{code}{Verbatim}{breaklines=true}

\usepackage{etoolbox}
\AtBeginEnvironment{code}{\linespread{1.0}\selectfont}
\usepackage{amsmath, amssymb}
\usepackage[font={normalfont}]{caption}
\captionsetup[figure]{name=Рисунок, labelsep=endash}

\setlength{\parskip}{0.5cm} %Отступ после абзаца
\linespread{1.5} %Междустрочнный интервал

\usepackage{enumitem}
\setlist[enumerate]{itemindent=\parskip}
\setlist[itemize]{itemindent=\parskip}

\setlength{\parindent}{1.5cm}

\titleformat{\section}
{\normalfont\bfseries}
{\thesection}{0.5em}{}

\titleformat{\subsection}
{\normalfont\bfseries\normalsize}
{\thesubsection}{0.5em}{}

\titlespacing{\section}{\parindent}{0pt}{0pt}
\titlespacing{\subsection}{\parindent}{0pt}{0pt}

\begin{document}
	
	\pagestyle{empty}
	
	\begin{center}
		
		\begin{onehalfspace}
			
			МИНИСТЕРСТВО НАУКИ И ВЫСШЕГО ОБРАЗОВАНИЯ \\
			РОССИЙСКОЙ ФЕДЕРАЦИИ \\
			
			\vspace{2mm}
			
			ФЕДЕРАЛЬНОЕ ГОСУДАРСТВЕННОЕ БЮДЖЕТНОЕ ОБРАЗОВАТЕЛЬНОЕ\\
			УЧРЕЖДЕНИЕ ВЫСШЕГО ОБРАЗОВАНИЯ\\
			«ВЯТСКИЙ ГОСУДАРСТВЕННЫЙ УНИВЕРСИТЕТ»\\
			
			\vspace{2mm}
			
			Институт математики и информационных систем\\
			\vspace{2mm}
			
			Факультет автоматики и вычислительной техники \\
			\vspace{2mm}
			
			Кафедра электронных вычислительных машин \\
			
		\end{onehalfspace}
		
	\end{center}
	
	\begin{flushright}
		\begin{small}
			Дата сдачи на проверку: \\
			«\underline{\hspace{5mm}}»
			\underline{\hspace{21mm}}
			2025 г. \\
			Проверка: \rule[0ex]{25mm}{0pt} \\
			«\underline{\hspace{5mm}}»
			\underline{\hspace{21mm}}
			2025 г. 
		\end{small}
	\end{flushright}
	
	\begin{center}
		
		
		\textbf{ЛАБОРАТОРНАЯ РАБОТА}
		
		Отчет по лабораторной работе №6 \\
		по дисциплине
		
		Вычислительная математика и методы оптимизаций
		
		\vspace{20mm}
		
		\begin{tabular}{ll}
			Разработал студент гр. ИВТб-2302-05-00 & \rule[0ex]{30mm}{0.1pt}~/Соловьев А.С./~ \\
			&  \rule[0ex]{8mm}{0pt}\scriptsize{(Подпись)} \\
			
			Проверил старший преподаватель &\rule[0ex]{30mm}{0.1pt}~/Коржавина А.С./~ \\
			&  \rule[0ex]{8mm}{0pt}\scriptsize{(Подпись)} \\
			
			Работа защищена & \hfill «\rule[0ex]{5mm}{0pt}»\rule[0ex]{24mm}{0.1pt}~2025г.~ 
			
		\end{tabular}
		
	\end{center}
	
	
	\vspace{8mm}
	\begin{center}
		Киров 2025
	\end{center}
	
	\newpage
	
	\pagestyle{plain}
	\setcounter{page}{2}
	

\section*{Дано (формулировка задачи)}

Рассматривается задача целочисленного линейного программирования:
\begin{equation*}
	\begin{aligned}
		&\text{Требуется:} \quad && \max \; z = c_1x_1 + c_2x_2 + \ldots + c_nx_n \\
		&\text{при условиях:} \quad &&
		\begin{cases}
			a_{11}x_1 + a_{12}x_2 + \ldots + a_{1n}x_n \leq b_1, \\
			a_{21}x_1 + a_{22}x_2 + \ldots + a_{2n}x_n \leq b_2, \\
			\quad \vdots \\
			a_{m1}x_1 + a_{m2}x_2 + \ldots + a_{mn}x_n \leq b_m, \\
			x_i \in \mathbb{Z}_{\geq 0}, \quad i = 1, \ldots, n.
		\end{cases}
	\end{aligned}
\end{equation*}

\section*{Алгоритм метода ветвей и границ}

\begin{enumerate}[label=\arabic*.]
	\item \textbf{Инициализация:}
	\begin{itemize}
		\item Начальная задача: \( (c, A, b) \)
		\item Лучшая целочисленная оценка: \( z^* = -\infty \)
		\item Лучшее решение: отсутствует
		\item Стек задач: \( \text{stack} = \{(c, A, b)\} \)
	\end{itemize}
	
	\item \textbf{Цикл по стеку задач:}
	\begin{itemize}
		\item Пока стек не пуст:
		\begin{itemize}
			\item Извлекаем текущую задачу \( (c, A, b) \)
			\item Решаем симплекс-методом (или двойственным симплекс-методом)
		\end{itemize}
	\end{itemize}
	
	\item \textbf{Анализ решения:}
	\begin{itemize}
		\item Если задача не имеет допустимого решения — отбрасываем ветку
		\item Если целевая функция не ограничена — отбрасываем ветку
		\item Если получено целочисленное решение:
		\begin{itemize}
			\item Если \( z > z^* \), обновляем \( z^* \) и решение
		\end{itemize}
		\item Если решение содержит дробные переменные:
		\begin{itemize}
			\item Выбираем первую нецелую переменную \( x_k \)
			\item Создаём две ветви:
			\begin{itemize}
				\item \( x_k \leq \lfloor x_k \rfloor \)
				\item \( x_k \geq \lceil x_k \rceil \)
			\end{itemize}
			\item Обе задачи добавляются в стек
		\end{itemize}
	\end{itemize}
	
	\item \textbf{Окончание:}
	\begin{itemize}
		\item Если стек пуст:
		\begin{itemize}
			\item Если найдено целочисленное решение — оно оптимально
			\item Иначе: целочисленного решения нет
		\end{itemize}
	\end{itemize}
\end{enumerate}

	
	
	
	\begin{figure}[H] 
		\center
		\includegraphics [width=0.8\linewidth]{Program.png}
		\caption{Часть работы программа в ответом}
	\end{figure} 
	
	
	
	\begin{center}
		\href{https://github.com/Akwil4ikekw/Optimization_labs.git}{Ссылка на репозиторий GitHub}
	\end{center}
	
	
	
	\section{Вывод}
	
	В ходе данной лабораторной работы была провелена деятельность по созданию алгоритма для реализации метода ветвей и границ.
	
\end{document}