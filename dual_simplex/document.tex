\documentclass[a4paper,12pt]{extarticle}
\usepackage[T1, T2A]{fontenc}
\usepackage{textcomp} %Спецсимволы
\usepackage{graphicx} %Графика
\usepackage{ragged2e} %Для использования {justify} - по ширине
\usepackage[english, russian]{babel}
\usepackage{setspace}
\usepackage{titlesec}
\usepackage[colorlinks=true,urlcolor=blue]{hyperref}  
\usepackage{float} % Для принудительного размещения float-объектов

\usepackage[a4paper, papersize={210mm, 297mm}, text={210mm, 297mm},
left=1.5cm, top=2cm, right=1.5cm, bottom=2cm]{geometry}

\usepackage{fvextra}
\DefineVerbatimEnvironment{code}{Verbatim}{breaklines=true}

\usepackage{etoolbox}
\AtBeginEnvironment{code}{\linespread{1.0}\selectfont}

\usepackage[font={normalfont}]{caption}
\captionsetup[figure]{name=Рисунок, labelsep=endash}

\setlength{\parskip}{0.5cm} %Отступ после абзаца
\linespread{1.5} %Междустрочнный интервал

\usepackage{enumitem}
\setlist[enumerate]{itemindent=\parskip}
\setlist[itemize]{itemindent=\parskip}

\setlength{\parindent}{1.5cm}

\titleformat{\section}
{\normalfont\bfseries}
{\thesection}{0.5em}{}

\titleformat{\subsection}
{\normalfont\bfseries\normalsize}
{\thesubsection}{0.5em}{}

\titlespacing{\section}{\parindent}{0pt}{0pt}
\titlespacing{\subsection}{\parindent}{0pt}{0pt}

\begin{document}
	
	\pagestyle{empty}
	
	\begin{center}
		
		\begin{onehalfspace}
			
			МИНИСТЕРСТВО НАУКИ И ВЫСШЕГО ОБРАЗОВАНИЯ \\
			РОССИЙСКОЙ ФЕДЕРАЦИИ \\
			
			\vspace{2mm}
			
			ФЕДЕРАЛЬНОЕ ГОСУДАРСТВЕННОЕ БЮДЖЕТНОЕ ОБРАЗОВАТЕЛЬНОЕ\\
			УЧРЕЖДЕНИЕ ВЫСШЕГО ОБРАЗОВАНИЯ\\
			«ВЯТСКИЙ ГОСУДАРСТВЕННЫЙ УНИВЕРСИТЕТ»\\
			
			\vspace{2mm}
			
			Институт математики и информационных систем\\
			\vspace{2mm}
			
			Факультет автоматики и вычислительной техники \\
			\vspace{2mm}
			
			Кафедра электронных вычислительных машин \\
			
		\end{onehalfspace}
		
	\end{center}
	
	\begin{flushright}
		\begin{small}
			Дата сдачи на проверку: \\
			«\underline{\hspace{5mm}}»
			\underline{\hspace{21mm}}
			2025 г. \\
			Проверка: \rule[0ex]{25mm}{0pt} \\
			«\underline{\hspace{5mm}}»
			\underline{\hspace{21mm}}
			2025 г. 
		\end{small}
	\end{flushright}
	
	\begin{center}
		
		
		\textbf{ЛАБОРАТОРНАЯ РАБОТА}
		
		Отчет по лабораторной работе №5 \\
		по дисциплине
		
		Вычислительная математика и методы оптимизаций
		
		\vspace{20mm}
		
		\begin{tabular}{ll}
			Разработал студент гр. ИВТб-2302-05-00 & \rule[0ex]{30mm}{0.1pt}~/Соловьев А.С./~ \\
			&  \rule[0ex]{8mm}{0pt}\scriptsize{(Подпись)} \\
			
			Проверил старший преподаватель &\rule[0ex]{30mm}{0.1pt}~/Коржавина А.С./~ \\
			&  \rule[0ex]{8mm}{0pt}\scriptsize{(Подпись)} \\
			
			Работа защищена & \hfill «\rule[0ex]{5mm}{0pt}»\rule[0ex]{24mm}{0.1pt}~2025г.~ 
			
		\end{tabular}
		
	\end{center}
	
	
	\vspace{8mm}
	\begin{center}
		Киров 2025
	\end{center}
	
	\newpage
	
	\pagestyle{plain}
	\setcounter{page}{2}
	
\section{Постановка задачи}
Дано: 
\begin{enumerate}
	\item Целевая функция задана в канонической форме: \textit{максимизировать} 
		\item Ограничения заданы в виде неравенств.
\end{enumerate}
	
\section{Алгоритм решения}
\begin{enumerate}
	\item Привести ограничения к канонической форме, добавив дополнительные переменные (например, добавочные или избыточные).
	\item Сформировать начальную симплекс-таблицу. Допускается, что в ней некоторые свободные члены будут отрицательными, но строка целевой функции должна быть оптимальной (все коэффициенты при переменных — неотрицательные).
	\item Найти строку с самым отрицательным свободным членом — она определяет ведущую строку (строка выхода).
	\item Среди отрицательных элементов в этой строке выбрать ведущий столбец (входную переменную) по правилу минимального отношения коэффициента в целевой функции к соответствующему элементу строки (с учетом правила Блэнда для предотвращения зацикливания).
	\item Выполнить симплекс-преобразование (привести ведущий элемент к единице, остальные — обнулить в столбце).
	\item Повторять шаги 3–5, пока все свободные члены не станут неотрицательными.
	\item Извлечь оптимальное решение и проверить его на допустимость и оптимальность.
\end{enumerate}


	
	\begin{figure}[H] 
		\center
		\includegraphics [width=0.8\linewidth]{Program.png}
		\caption{Результат работы программа }
	\end{figure} 
	
	
	
	\begin{center}
		\href{https://github.com/Akwil4ikekw/Optimization_labs.git}{Ссылка на репозиторий GitHub}
	\end{center}
	
	
	
	\section{Вывод}
	
	В ходе данной лабораторной работы была провелена деятельность по созданию алгоритма для реализации двойственного симплекса.
	
\end{document}