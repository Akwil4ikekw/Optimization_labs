\documentclass[a4paper,12pt]{extarticle}
\usepackage[T1, T2A]{fontenc}
\usepackage{textcomp} %Спецсимволы
\usepackage{graphicx} %Графика
\usepackage{ragged2e} %Для использования {justify} - по ширине
\usepackage[english, russian]{babel}
\usepackage{setspace}
\usepackage{titlesec}
\usepackage[colorlinks=true,urlcolor=blue]{hyperref}  
\usepackage{float} % Для принудительного размещения float-объектов
\usepackage{amsmath}        % Для математических сред и команд
\usepackage{booktabs}       % Для красивых горизонтальных линий в таблицах (\toprule, \midrule, \bottomrule)
\usepackage{siunitx} 
\usepackage[a4paper, papersize={210mm, 297mm}, text={210mm, 297mm},
left=1.5cm, top=2cm, right=1.5cm, bottom=2cm]{geometry}

\usepackage{fvextra}
\DefineVerbatimEnvironment{code}{Verbatim}{breaklines=true}

\usepackage{etoolbox}
\AtBeginEnvironment{code}{\linespread{1.0}\selectfont}
\usepackage{amsmath, amssymb}
\usepackage[font={normalfont}]{caption}
\captionsetup[figure]{name=Рисунок, labelsep=endash}

\setlength{\parskip}{0.5cm} %Отступ после абзаца
\linespread{1.5} %Междустрочнный интервал

\usepackage{enumitem}
\setlist[enumerate]{itemindent=\parskip}
\setlist[itemize]{itemindent=\parskip}

\setlength{\parindent}{1.5cm}

\titleformat{\section}
{\normalfont\bfseries}
{\thesection}{0.5em}{}

\titleformat{\subsection}
{\normalfont\bfseries\normalsize}
{\thesubsection}{0.5em}{}

\titlespacing{\section}{\parindent}{0pt}{0pt}
\titlespacing{\subsection}{\parindent}{0pt}{0pt}

\begin{document}
	
	\pagestyle{empty}
	
	\begin{center}
		
		\begin{onehalfspace}
			
			МИНИСТЕРСТВО НАУКИ И ВЫСШЕГО ОБРАЗОВАНИЯ \\
			РОССИЙСКОЙ ФЕДЕРАЦИИ \\
			
			\vspace{2mm}
			
			ФЕДЕРАЛЬНОЕ ГОСУДАРСТВЕННОЕ БЮДЖЕТНОЕ ОБРАЗОВАТЕЛЬНОЕ\\
			УЧРЕЖДЕНИЕ ВЫСШЕГО ОБРАЗОВАНИЯ\\
			«ВЯТСКИЙ ГОСУДАРСТВЕННЫЙ УНИВЕРСИТЕТ»\\
			
			\vspace{2mm}
			
			Институт математики и информационных систем\\
			\vspace{2mm}
			
			Факультет автоматики и вычислительной техники \\
			\vspace{2mm}
			
			Кафедра электронных вычислительных машин \\
			
		\end{onehalfspace}
		
	\end{center}
	
	\begin{flushright}
		\begin{small}
			Дата сдачи на проверку: \\
			«\underline{\hspace{5mm}}»
			\underline{\hspace{21mm}}
			2025 г. \\
			Проверка: \rule[0ex]{25mm}{0pt} \\
			«\underline{\hspace{5mm}}»
			\underline{\hspace{21mm}}
			2025 г. 
		\end{small}
	\end{flushright}
	
	\begin{center}
		
		
		\textbf{ЛАБОРАТОРНАЯ РАБОТА}
		
		Отчет по лабораторной работе №9 \\
		по дисциплине
		
		Вычислительная математика и методы оптимизаций
		
		\vspace{20mm}
		
		\begin{tabular}{ll}
			Разработал студент гр. ИВТб-2302-05-00 & \rule[0ex]{30mm}{0.1pt}~/Соловьев А.С./~ \\
			&  \rule[0ex]{8mm}{0pt}\scriptsize{(Подпись)} \\
			
			Проверил старший преподаватель &\rule[0ex]{30mm}{0.1pt}~/Старостин П.А./~ \\
			&  \rule[0ex]{8mm}{0pt}\scriptsize{(Подпись)} \\
			
			Работа защищена & \hfill «\rule[0ex]{5mm}{0pt}»\rule[0ex]{24mm}{0.1pt}~2025г.~ 
			
		\end{tabular}
		
	\end{center}
	
	
	\vspace{8mm}
	\begin{center}
		Киров 2025
	\end{center}
	
	\newpage
	
	\pagestyle{plain}
	\setcounter{page}{2}
	
	\section{Задание 1}
	\subsection{Условие}
	С помощью квадратурной формулы Гаусса и 5 известных точек, вычислить интеграл от полинома \[\int_{-1}^{2} x^{8}+5x^{5}+2x^{2} \] с алгебраической точностью 9 и 7. Сравнить получившиеся результаты с аналитическим решением. В отчет по работе должны включаться:
	\begin{enumerate}
		\item График общей функции полинома
		\item График на оперделенном интервале 
		\item Расчет аналитеским способом интеграла
		\item Расчет методом Гаусса
		\item Сравнить аналитическое решение при условии, что полином будет превышать алгебраическую точность
	\end{enumerate}
	\subsection{Решение }
	Формула Гаусса:
	\[
	\int_a^b f(x) \, dx \approx \frac{b-a}{2} \sum_{i=1}^n w_i f\left( \frac{b-a}{2} x_i + \frac{b+a}{2} \right)
	\]
	
	где:
	\begin{itemize}
		\item $n$ -- количество узлов (точек) для квадратуры.
		\item $x_i$ -- $i$-й узел Гаусса.
		\item $w_i$ -- вес, соответствующий $i$-му узлу $x_i$.
	
	Построим график функции на интервале $x\in [-100;100]$
	
		
	\begin{figure}[H] 
		\center
		\includegraphics [width=0.6\linewidth]{PolinomMain.jpg}
		\caption{График функции}
	\end{figure} 
	
	Построим график функции на интервале, на котором будем делать интегрирование $x\in [-1;2]$
	
	
	\begin{figure}[H] 
		\center
		\includegraphics [width=0.6\linewidth]{PolinomOnInterval.jpg}
		\caption{График функции на интегрируемом интервале}
	\end{figure} 
	Рассчитаем интеграл аналитическим методом:
	\[f(x) = x^{8}+5x^{5}+2x^{2}	\]
	\[\int_{-1}^{2} x^{8}+5x^{5}+2x^{2} = \left. \frac{x^{9}}{9} + \frac{5x^{6}}{6} + \frac{2x^{3}}{3} \right|_{-1}^{2}\]
	В результате вычислений интеграл был получен аналитическим способом:
	\[\int_{-1}^{2} x^{8}+5x^{5}+2x^{2} = 115\frac{1}{2}\]
	
	Расчитаем интеграл с помощью программы написанной в Python. Сначала вычислим интеграл с алгебраической точностью 9, для этого используем 5 точек.
	
	\begin{tabular}{|c|c|}
		\hline
		($x_i$) & Вес ($w_i$) \\
		\hline
		-0.90617985 & 0.23692689 \\
		-0.53846931 & 0.47862867 \\
		0.0         & 0.56888889 \\
		0.53846931  & 0.47862867 \\
		0.90617985  & 0.23692689 \\
		\hline
	\end{tabular}
	
		\begin{figure}[H] 
		\center
		\includegraphics [width=0.9\linewidth]{5Point.png}
		\caption{Значение интеграла методом Гаусса для точности 9}
	\end{figure} 
	
	В результате работы программы получился ответ
	\\ I$\approx 115.50$
	\\Как видно погрешность для 5 узлов очень мала и стремится к 0.
	
	Расчитаем с точностью, которая меньше степени полнома. Возьмем алгебраическую точность 7. То есть нам нужно 4 точки ($2n-1$).
	
			\begin{figure}[H] 
		\center
		\includegraphics [width=0.9\linewidth]{4Point.png}
		\caption{Значение интеграла методом Гаусса для точности 7}
	\end{figure} 
	
	Результате работы программы ответ получился:
	\\	\\ I$\approx 115.05$
	\\Видно, что погрешность $e \approx 0.45$ в случае, когда полином превосходит алгебраическую точность допустимую, то выходит погрешность по сравненю с аналитическим вычислением.
	
	
	\section{Задание 2}
	\subsection{Условие}
	Решить дифференциальное уравнение с помощью метода Рунге-Кутта 2го порядка. Сравнить график исходной функции и полученной в результате решения дифференциального уравнения.
	\\ Исходная функция:
	\[y(x) = x^{6} + x^{5} + 8\]
	Функция производной:
	\[ y' = 6x^{5} + 5x^{4}\]
	с начальным условием:
	\[y(0) = 8\]
	\subsection{Решение}
	
	
	
	
	\subsubsection*{Таблица результатов для шага $h=0.1$}
	\begin{table}[h!]
		\centering
		\caption{Результаты метода Рунге-Кутты (РК) для шага $h=0.1$}
		\label{tab:rk_h01_new} % Новая метка для избежания дублирования
		\begin{tabular}{
				S[table-format=1.1]
				S[table-format=1.6] % Учитываем, что y(approx) начинается с 8.0
				S[table-format=1.6] % Учитываем, что y(true) начинается с 6.0
				S[table-format=1.6e-1] % Ошибка начинается с 2.0e+00
			}
			\toprule
			{$x$} & {y(прибл.)} & {y(точное)} & {Ошибка} \\
			\midrule
			0.0 & 8.000000 & 8.000000 & 0.000000e+00 \\ % ИСПРАВЛЕНО
			0.1 & 8.000028 & 8.000011 & 1.700000e-05 \\ % Приблизительное значение ошибки после исправления true_func
			0.2 & 8.000552 & 8.000384 & 1.680000e-04 \\
			0.3 & 8.003802 & 8.003159 & 6.430000e-04 \\
			0.4 & 8.016028 & 8.014336 & 1.692000e-03 \\
			0.5 & 8.050500 & 8.046875 & 3.625000e-03 \\
			0.6 & 8.131228 & 8.124416 & 6.812000e-03 \\
			0.7 & 8.297402 & 8.285719 & 1.168300e-02 \\
			0.8 & 8.608552 & 8.589824 & 1.872800e-02 \\
			0.9 & 9.150428 & 9.121931 & 2.849700e-02 \\
			1.0 & 10.041600 & 10.000000 & 4.160000e-02 \\
			1.1 & 11.440778 & 11.382071 & 5.870700e-02 \\
			1.2 & 13.554852 & 13.474304 & 8.054800e-02 \\
			1.3 & 16.647652 & 16.539739 & 1.079130e-01 \\
			1.4 & 21.049428 & 20.907776 & 1.416520e-01 \\
			1.5 & 27.167050 & 26.984375 & 1.826750e-01 \\
			1.6 & 35.494928 & 35.262976 & 2.319520e-01 \\
			1.7 & 46.626652 & 46.336139 & 2.905130e-01 \\
			1.8 & 61.267352 & 60.907904 & 3.594480e-01 \\
			1.9 & 80.246778 & 79.806871 & 4.399070e-01 \\
			2.0 & 104.533100 & 104.000000 & 5.331000e-01 \\
			\bottomrule
		\end{tabular}
	\end{table}
	
	\subsubsection*{Таблица результатов для шага $h=0.05$}
	\begin{table}[h!]
		\centering
		\caption{Результаты метода Рунге-Кутты (РК) для шага $h=0.05$}
		\label{tab:rk_h005_new} % Новая метка для избежания дублирования
		\begin{tabular}{
				S[table-format=1.1]
				S[table-format=1.6]
				S[table-format=1.6]
				S[table-format=1.6e-1]
			}
			\toprule
			{$x$} & {y(прибл.)} & {y(точное)} & {Ошибка} \\
			\midrule
			0.0 & 8.000000 & 8.000000 & 0.000000e+00 \\
			0.1 & 8.000016 & 8.000011 & 5.000000e-06 \\
			0.2 & 8.000427 & 8.000384 & 4.300000e-05 \\
			0.3 & 8.003322 & 8.003159 & 1.630000e-04 \\
			0.4 & 8.014762 & 8.014336 & 4.260000e-04 \\
			0.5 & 8.047785 & 8.046875 & 9.100000e-04 \\
			0.6 & 8.126124 & 8.124416 & 1.708000e-03 \\
			0.7 & 8.288647 & 8.285719 & 2.928000e-03 \\
			0.8 & 8.594515 & 8.589824 & 4.691000e-03 \\
			0.9 & 9.129066 & 9.121931 & 7.135000e-03 \\
			1.0 & 10.010413 & 10.000000 & 1.041300e-02 \\
			1.1 & 11.396763 & 11.382071 & 1.469200e-02 \\
			1.2 & 13.494458 & 13.474304 & 2.015400e-02 \\
			1.3 & 16.566737 & 16.539739 & 2.699800e-02 \\
			1.4 & 20.943212 & 20.907776 & 3.543600e-02 \\
			1.5 & 27.030070 & 26.984375 & 4.569500e-02 \\
			1.6 & 35.320993 & 35.262976 & 5.801700e-02 \\
			1.7 & 46.408800 & 46.336139 & 7.266100e-02 \\
			1.8 & 60.997802 & 60.907904 & 8.989800e-02 \\
			1.9 & 79.916888 & 79.806871 & 1.100170e-01 \\
			2.0 & 104.133319 & 104.000000 & 1.333190e-01 \\
			\bottomrule
		\end{tabular}
	\end{table}
	
			\begin{figure}[H] 
		\center
		\includegraphics [width=0.9\linewidth]{Example2.png}
		\caption{Графики функций}
	\end{figure} 
	
	На графике изображена истинная функция пунктиром и функции, полученные в результате решение дифференциального уравнения с разными шагами. Функции очень близки и почти совпадают с минимальной погрешностью.
	
\end{document}