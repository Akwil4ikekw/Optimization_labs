\documentclass[a4paper,14pt]{extarticle}
\usepackage[T1, T2A]{fontenc}
\usepackage{textcomp}

\usepackage[utf8]{inputenc}
\usepackage{listings}
\usepackage{xcolor}  % Для цветов в коде

\lstset{
	language=Python,
	basicstyle=\ttfamily\footnotesize,
	numbers=left,
	numberstyle=\tiny\color{gray},
	stepnumber=1,
	numbersep=5pt,
	backgroundcolor=\color{white},
	showspaces=false,
	showstringspaces=false,
	showtabs=false,
	frame=single,
	rulecolor=\color{black},
	tabsize=2,
	captionpos=b,
	breaklines=true,
	breakatwhitespace=false,
	escapeinside={\%*}{*)},
	keywordstyle=\color{blue},
	commentstyle=\color{green},
	stringstyle=\color{red},
	literate={а}{{\selectfont\char224}}1
	{б}{{\selectfont\char225}}1
	{в}{{\selectfont\char226}}1
	{г}{{\selectfont\char227}}1
	{д}{{\selectfont\char228}}1
	{е}{{\selectfont\char229}}1
	{ё}{{\selectfont\char184}}1
	{ж}{{\selectfont\char230}}1
	{з}{{\selectfont\char231}}1
	{и}{{\selectfont\char232}}1
	{й}{{\selectfont\char233}}1
	{к}{{\selectfont\char234}}1
	{л}{{\selectfont\char235}}1
	{м}{{\selectfont\char236}}1
	{н}{{\selectfont\char237}}1
	{о}{{\selectfont\char238}}1
	{п}{{\selectfont\char239}}1
	{р}{{\selectfont\char240}}1
	{с}{{\selectfont\char241}}1
	{т}{{\selectfont\char242}}1
	{у}{{\selectfont\char243}}1
	{ф}{{\selectfont\char244}}1
	{х}{{\selectfont\char245}}1
	{ц}{{\selectfont\char246}}1
	{ч}{{\selectfont\char247}}1
	{ш}{{\selectfont\char248}}1
	{щ}{{\selectfont\char249}}1
	{ъ}{{\selectfont\char250}}1
	{ы}{{\selectfont\char251}}1
	{ь}{{\selectfont\char252}}1
	{э}{{\selectfont\char253}}1
	{ю}{{\selectfont\char254}}1
	{я}{{\selectfont\char255}}1
	{А}{{\selectfont\char192}}1
	{Б}{{\selectfont\char193}}1
	{В}{{\selectfont\char194}}1
	{Г}{{\selectfont\char195}}1
	{Д}{{\selectfont\char196}}1
	{Е}{{\selectfont\char197}}1
	{Ё}{{\selectfont\char168}}1
	{Ж}{{\selectfont\char198}}1
	{З}{{\selectfont\char199}}1
	{И}{{\selectfont\char200}}1
	{Й}{{\selectfont\char201}}1
	{К}{{\selectfont\char202}}1
	{Л}{{\selectfont\char203}}1
	{М}{{\selectfont\char204}}1
	{Н}{{\selectfont\char205}}1
	{О}{{\selectfont\char206}}1
	{П}{{\selectfont\char207}}1
	{Р}{{\selectfont\char208}}1
	{С}{{\selectfont\char209}}1
	{Т}{{\selectfont\char210}}1
	{У}{{\selectfont\char211}}1
	{Ф}{{\selectfont\char212}}1
	{Х}{{\selectfont\char213}}1
	{Ц}{{\selectfont\char214}}1
	{Ч}{{\selectfont\char215}}1
	{Ш}{{\selectfont\char216}}1
	{Щ}{{\selectfont\char217}}1
	{Ъ}{{\selectfont\char218}}1
	{Ы}{{\selectfont\char219}}1
	{Ь}{{\selectfont\char220}}1
	{Э}{{\selectfont\char221}}1
	{Ю}{{\selectfont\char222}}1
	{Я}{{\selectfont\char223}}1
}

\usepackage[normalem]{ulem} 
\usepackage{tabularray}
\usepackage{graphicx}
\usepackage{diagbox} 
\usepackage{multirow}
\usepackage{float}
\usepackage{wrapfig}
\usepackage{ragged2e}
\usepackage[english, russian]{babel}
\usepackage{setspace}
\setlength{\parskip}{0.5cm}
\linespread{1.5}
\setlength{\parindent}{0.5cm}
\usepackage[a4paper, papersize={210mm, 297mm}, text={210mm, 297mm},
left=2cm, top=2cm, right=1.5cm, bottom=2cm]{geometry}
\begin{document} 
	\begin{center}
		\normalsize{МИНИСТЕРСТВО НАУКИ И ВЫСШЕГО ОБРАЗОВАНИЯ
			РОССИЙСКОЙ ФЕДЕРАЦИИ}\\
		\normalsize{ФЕДЕРАЛЬНОЕ ГОСУДАРСТВЕННОЕ БЮДЖЕТНОЕ ОБРАЗОВАТЕЛЬНОЕ
			УЧРЕЖДЕНИЕ ВЫСШЕГО ОБРАЗОВАНИЯ\\
			«ВЯТСКИЙ ГОСУДАРСТВЕННЫЙ УНИВЕРСИТЕТ»}\\ 
		\normalsize{Институт математики и информационных систем}\\
		\normalsize{Факультет автоматики и вычислительной техники}\\
		\normalsize{Кафедра электронных вычислительных машин}\\
	\end{center}
	
	\begin{flushright}
		\small{Дата сдачи на проверку:}\\
		\small{«}\underline{\hspace{0.5cm}}\small{»}\underline{\hspace{2cm}} \small{2025 г.}\\
		\small{Проверено:} \rule[0ex]{2.3cm}{0pt}\\
		\small{«}\underline{\hspace{0.5cm}}\small{»}\underline{\hspace{2cm}}\small{2025 г.} 
	\end{flushright}
	\begin{center}
		\normalsize{Вариант 18}\\
		\normalsize{Отчет по лабораторной работе № 4 }\\ по дисциплине \\ 
		\normalsize{«Вычислительная математика»}\\
		
		
	\end{center}
	
	
	\begin{flushleft}
		\hfill \break
		\hfill \break
		\normalsize{ Разработал студент гр. ИВТб 2302-05-00}\rule[0ex]{0.1cm}{0pt} \underline{\hspace{4cm}}  \normalsize{/Соловьев А.С./} \\
		\rule[0ex]{9.6cm}{0pt} \tiny{(Подпись)}\\
		\normalsize{ Проверил заведующий кафедры ЭВМ}  \rule[0ex]{0.1cm}{0pt} \underline{\hspace{4cm}} \normalsize{/Старостин П.А./} \\ 
		\rule[0ex]{9.6cm}{0pt} \tiny{(Подпись)}\\
		\normalsize{ Работа защищена} \rule[0ex]{8.5cm}{0pt} \small{«}\underline{\hspace{0.5cm}}\small{»}\underline{\hspace{2.6cm}}\small{2025 г.}
		\hfill \break
		\hfill \break
	\end{flushleft}
	\begin{center} Киров 2025 \end{center}
	\thispagestyle{empty} % выключаем отображение номера для этой страницы
	
	\newpage
	
	\section{Задание}
	\begin{enumerate}
		\item Вычислить определенный интеграл с точностью до 0.0001. Выбрать значение n обеспечивающее заданную точность из формулы остатосного члена.\\
		Приведенный интеграл функции $\frac{1}{\sqrt{x^2+1.2}}$\\
		Пределы интегрирования $x \in [1.2;2.0]$\\
		Использовать формулу трапеций\\
	\item Вычислить опреденный интеграл с точность до 0.0001 по другой квадратурной формуле:\\
	Задание:
	\\Определнный интеграл от функции $\frac{lg(x^{2}+3)}{2x}$
	Пределы интегрирования $x\in [1.2;2.0]$
	\item Вычислить опреденный интеграл квадратурной формуле Гаусса. Для оценки погрешности взять различное количество узлов:
	\\$n_1 = 4$;$n_2 = 7$\\
	Квадратичная формула Гаусса с 4 узлами:
	
	
	$x_1 = x_4 = -0.86114$\\
	$x_2 = x_3 = -0.33998$\\
	$A_1 = A_4 = 0.34785$\\
	$A_2 = A_3 = 0.65215$\\	
	 
	Квадратурная формула Гаусса с 7 узлами:
	
	$x_1 = x_7 = -0.949107912$\\
	$x_2 = x_6 = 0.741531186$\\
	$x_3 = x_5 = 0.405845151$\\
	$x_4 = 0$\\
	$A_1 = A_7 = 0.129484966$\\
	$A_2 = A_6 = 0.279705391$\\ 
	$A_3 = A_5 = 0.38183005$\\
	$A_4 = 0.417959184$\\
	\item Определить значения всех интегралов, обратившись к втроенным функциям Mathcad или к аналогам для проверки вычислений.
	\item Решить обыкновенное дифференциальное уравнение. Решение представить в табличной и графических формах. Для оценки погрешности выполнить расчет с шагом h и $h/2$
	\end{enumerate}
	
	

\section{Метод с формулой трапеций}

Метод трапеций — это численный метод приближённого вычисления определённого интеграла. Для функции $f(x)$, заданной на отрезке $[a, b]$, интеграл приближается по формуле:

\[
\int_a^b f(x) \,dx \approx \frac{h}{2} \left( f(a) + f(b) \right) + h \sum_{i=1}^{n-1} f(a + i h)
\]

где $h = \frac{b - a}{n}$ — шаг разбиения, $n$ — количество отрезков разбиения.

\subsection{Оценка погрешности}

Погрешность метода трапеций можно оценить двумя способами:

\subsubsection{Через вторую производную (априорная оценка)}

Если функция $f(x)$ имеет непрерывную вторую производную на $[a, b]$, то погрешность $R_n$ метода трапеций оценивается как:

\[
|R_n| \leq \frac{(b - a)^3}{12 n^2} \max_{x \in [a, b]} |f''(x)|
\]

\begin{itemize}
	\item Преимущество: позволяет заранее определить необходимое $n$ для заданной точности $\epsilon$
	\item Недостаток: требует знания и оценки второй производной
	\item Применение: вычисляется $n = \left\lceil \sqrt{\frac{(b - a)^3 \max |f''|}{12 \epsilon}} \right\rceil$
\end{itemize}

\subsubsection{Принцип Рунге (апостериорная оценка)}

Погрешность оценивается путём сравнения результатов при разных $n$:

\[
|I - I_n| \approx \frac{|I_{2n} - I_n|}{3}
\]

где $I_n$ — значение интеграла при $n$ отрезках.

\begin{itemize}
	\item Преимущество: не требует знания производных, автоматически адаптируется к функции
	\item Недостаток: требует дополнительных вычислений
	\item Алгоритм:
	\begin{enumerate}
		\item Начинаем с малого $n$ (например, $n=2$)
		\item Удваиваем $n$ пока $\frac{|I_{2n} - I_n|}{3} > \epsilon$
		\item Результат — $I_{2n}$ с погрешностью $\leq \epsilon$
	\end{enumerate}
\end{itemize}

\subsection{Сравнение подходов}

\begin{itemize}
	\item Для гладких функций с известными производными эффективен первый метод
	\item Для функций с неизвестным поведением производных предпочтителен второй метод
	\item На практике часто используют комбинацию: начинают с оценки производной, затем уточняют по Рунге
\end{itemize}

\section{Метод Симпсона для численного интегрирования}

Метод Симпсона — это численный метод приближённого вычисления определённого интеграла, основанный на квадратичной интерполяции подынтегральной функции.

\subsection{Основная формула}

Для функции $f(x)$ на отрезке $[a,b]$ интеграл приближается по формуле:

\[
\int_a^b f(x)dx \approx \frac{h}{3} \left[ f(a) + 4f\left(\frac{a+b}{2}\right) + f(b) \right]
\]

где $h = \frac{b-a}{2}$ — шаг интегрирования.

Для составного варианта (при разбиении на $n$ чётное число отрезков):

\[
\int_a^b f(x)dx \approx \frac{h}{3} \left[ f(x_0) + 4\sum_{i=1}^{n/2} f(x_{2i-1}) + 2\sum_{i=1}^{n/2-1} f(x_{2i}) + f(x_n) \right]
\]

\subsection{Происхождение формулы}

Формула Симпсона получается при:
\begin{enumerate}
	\item Разбиении отрезка $[a,b]$ на чётное число интервалов
	\item Аппроксимации функции на каждой паре соседних интервалов квадратичным полиномом (параболой)
	\item Точном интегрировании этого полинома
\end{enumerate}

Метод имеет порядок точности $O(h^4)$ для одиночного отрезка и $O(h^5)$ для составной формулы.

\subsection{Оценка погрешности}

Погрешность метода Симпсона можно оценить двумя способами:

\subsubsection{ Принцип Рунге (апостериорная оценка)}

Практическая оценка погрешности:

\[
|I - I_n| \approx \frac{|I_{2n} - I_n|}{15}
\]

где $I_n$ — значение интеграла при $n$ отрезках.

\subsection{Алгоритм адаптивного интегрирования}

\begin{enumerate}
	\item Начать с минимального чётного $n$ (обычно 2 или 4)
	\item Вычислить $I_n$ и $I_{2n}$
	\item Оценить погрешность $\varepsilon = \frac{|I_{2n} - I_n|}{15}$
	\item Если $\varepsilon > \text{требуемая точность}$, удвоить $n$ и повторить
	\item Результат — $I_{2n}$ с гарантированной точностью
\end{enumerate}


\section{Квадратурные формулы Гаусса}

\subsection{Основная идея метода}
Квадратурные формулы Гаусса — это численные методы интегрирования вида:

\[
\int_{-1}^1 f(x)dx \approx \sum_{i=1}^n w_i f(x_i)
\]

где:
\begin{itemize}
	\item $x_i$ — узлы (корни многочленов Лежандра)
	\item $w_i$ — весовые коэффициенты
	\item $n$ — количество узлов
\end{itemize}

\subsection{Перенос на произвольный интервал}
Для интегрирования на произвольном интервале $[a,b]$ используется замена переменных:

\[
\int_a^b f(x)dx = \frac{b-a}{2} \int_{-1}^1 f\left(\frac{b-a}{2}t + \frac{a+b}{2}\right)dt
\]



%\subsection{Таблицы узлов и весов}
%\begin{center}
	%\begin{tabular}{|c|c|c|}
%		\hline
%		$n$ & Узлы $x_i$ & Веса $w_i$ \\ \hline
%		4 & $\pm0.86114$, $\pm0.33998$ & $0.34785$, $0.65215$ \\ \hline
%		7 & $\pm0.94911$, $\pm0.74153$, $\pm0.40585$, $0.0$ & $0.12948$, $0.27971$, $0.38183$, $0.41796$ \\ \hline
%	\end{tabular}
%\end{center}



\subsection{Сравнение точности}
\begin{itemize}
	\item Для $n=4$: точность до 7-й степени полинома
	\item Для $n=7$: точность до 13-й степени полинома
	\item Погрешность оценивается по формуле:
	\[
	E_n = \frac{f^{(2n)}(\xi)}{(2n)!} \int_{-1}^1 [P_n(x)]^2 dx
	\]
	где $\xi \in (-1,1)$
\end{itemize}


\section{Метод Рунге-Кутты 2-го порядка для решения ОДУ}

\subsection{Постановка задачи}
Дано обыкновенное дифференциальное уравнение (ОДУ) первого порядка:
\[
\frac{dy}{dx} = f(x, y), \quad y(x_0) = y_0
\]
Требуется найти приближенное решение $y(x)$ на интервале $[x_0, x_{end}]$.

\subsection{Метод Рунге-Кутты 2-го порядка}
Модифицированный метод Эйлера (один из вариантов РК2):

\begin{enumerate}
	\item Выбираем шаг $h$
	\item На каждом шаге вычисляем:\\
	$	k_1 = h \cdot f(x_n, y_n) $\\
	$	k_2 = h \cdot f(x_n + h, y_n + k_1) $\\
	$	y_{n+1} = y_n + \frac{1}{2}(k_1 + k_2)$\\
	\item Переходим к следующей точке: $x_{n+1} = x_n + h$
\end{enumerate}

\subsection{Геометрическая интерпретация}
\begin{itemize}
	\item $k_1$ - наклон в начальной точке отрезка
	\item $k_2$ - наклон в конечной точке (с учётом приближения по Эйлеру)
	\item Результирующий наклон - среднее этих двух значений
\end{itemize}


\subsection{Погрешность метода}
\begin{itemize}
	\item Локальная погрешность: $O(h^3)$
	\item Глобальная погрешность: $O(h^2)$ (метод второго порядка)
	\item Для оценки погрешности часто используют правило Рунге:
	\[
	\varepsilon \approx \frac{|y_h - y_{h/2}|}{3}
	\]
\end{itemize}

\subsection{Что является решением дифференциального уравнения?}
Решение дифференциального уравнения - это:
\begin{itemize}
	\item Функция $y(x)$, удовлетворяющая уравнению во всех точках области определения
	\item В численных методах - таблица приближенных значений $\{x_i, y_i\}$
	\item График интегральной кривой в плоскости $(x, y)$
	\item Процесс последовательного приближения к точному решению при уменьшении шага
\end{itemize}

	\section{Задание 1}

	
	\subsection{Постановка задачи}
	Требуется вычислить определённый интеграл функции 
	\[ f(x) = \dfrac{1}{\sqrt{x^2 + 1.2}} \]
	на отрезке $[1.2, 2.0]$ с точностью $\varepsilon = 0.0001$ методом трапеций.
	
	\subsection{Теоретическая основа}
	Метод трапеций вычисляет интеграл по формуле:
	\[
	\int_a^b f(x)dx \approx \frac{h}{2}\left(f(a) + f(b)\right) + h\sum_{i=1}^{n-1} f(a + ih)
	\]
	где $h = \frac{b-a}{n}$ --- шаг разбиения.
	
	Погрешность метода оценивается как:
	\[
	|R_n| \leq \frac{(b-a)^3}{12n^2} \max_{x\in[a,b]} |f''(x)|
	\]
	
	\subsection{Решение}
	
	\subsubsection{Способ 1: Через оценку второй производной}
	1. Найдём вторую производную функции:
	\[
	f''(x) = \frac{3x^2}{(x^2 + 1.2)^{5/2}} - \frac{1}{(x^2 + 1.2)^{3/2}}
	\]
	
	2. Найдём максимум $|f''(x)|$ на $[1.2, 2.0]$ численно, перебирая 1000 точек.
	
	3. По формуле оценки погрешности определяем необходимое число разбиений:
	\[
	n \geq \sqrt{\frac{(b-a)^3 \max|f''(x)|}{12\varepsilon}} = \sqrt{\frac{0.8^3 \cdot \max|f''(x)|}{12 \cdot 0.0001}}
	\]
	
	4. Вычисляем интеграл с найденным $n$.
	
	Результат:
	\begin{itemize}
		\item Число разбиений: $n = 83$
		\item Значение интеграла: $0.379888$
		\item Оценка погрешности: $0.000099$
	\end{itemize}
	
	\subsubsection{Способ 2: Принцип Рунге}
	1. Начинаем с $n=2$ и удваиваем число разбиений, пока оценка погрешности не станет меньше $\varepsilon$.
	
	2. Оценка погрешности по Рунге для метода трапеций:
	\[
	|I - I_n| \approx \frac{|I_{2n} - I_n|}{3}
	\]
	
	3. Процесс продолжается до выполнения условия:
	\[
	\frac{|I_{2n} - I_n|}{3} < \varepsilon
	\]
	
	\begin{figure}[H] 
		\center
		\includegraphics[width=0.8\textwidth]{Задание1Результат.png}
	\caption{Результат работы программы} 
	\end{figure} 
	
\begin{figure}[H] 
	\center
		\includegraphics[width=0.8\textwidth]{Задание1График.png}
		\caption{Исходный график функции} 
	\end{figure}
		График показывает поведение интегрируемой функции на заданном интервале. Функция монотонно убывает, что подтверждает корректность применения метода трапеций.
	
	
	\begin{figure}[H] 
		\center
		\includegraphics[width=0.8\textwidth]{Задание1Проверка.png}
		\caption{Проверка результата работы программы} 
	\end{figure}
	Результаты работы программы сошлись с проверкой из другого ресурса.
	

	\subsection{Выводы}
	Оба метода дали близкие результаты:
	\begin{itemize}
		\item Через оценку производной: $0.415668$
		\item Через принцип Рунге: $0.415965$
	\end{itemize}
	
	Разница между результатами составляет $0.000001$, что меньше требуемой точности. Принцип Рунге оказался более эффективным, так как потребовал меньшего числа разбиений (9 против 4) для достижения той же точности
\subsection*{Реализация на Python}
\lstinputlisting[language=Python, caption={Python код программы}]{Laba4_1.py}
\section{Задача 2}
\subsection{Постановка задачи}
Требуется вычислить определённый интеграл функции 
\[ f(x) = \frac{\log_{10}(x^2 + 3)}{2x} \]
на отрезке $[1.2, 2.0]$ с точностью $\varepsilon = 0.0001$ методом Симпсона.

\subsection{Теоретическая основа}
Метод Симпсона вычисляет интеграл по формуле:
\[
\int_a^b f(x)dx \approx \frac{h}{3}\left[f(a) + 4\sum_{\text{нечётные } i} f(x_i) + 2\sum_{\text{чётные } i} f(x_i) + f(b)\right]
\]
где $h = \frac{b-a}{n}$ --- шаг разбиения ($n$ должно быть чётным).

Оценка погрешности по правилу Рунге:
\[
|R_n| \approx \frac{|I_{2n} - I_n|}{15}
\]

\subsection{Решение}

\subsubsection{Адаптивный алгоритм}
1. Начинаем с $n=4$ (гарантированно чётное)
2. Последовательно удваиваем число разбиений до достижения точности
3. На каждой итерации оцениваем погрешность по Рунге
4. Процесс прекращается при выполнении условия:
\[
\frac{|I_{2n} - I_n|}{15} < \varepsilon
\]

\subsubsection{Ход вычислений}
\begin{center}
	\begin{tabular}{cccc}
		\toprule
		Итерация & $n$ & Значение интеграла & Оценка погрешности \\
		\midrule
		1 & 8 & 0.22058042 & 1.05e-04 \\
		2 & 16 & 0.22047794 & 6.83e-06 \\
		3 & 32 & 0.22047111 & 4.55e-07 \\
		\bottomrule
	\end{tabular}
\end{center}

\subsection{Результаты}
\begin{itemize}
	\item Итоговое число разбиений: $n = 8$
	\item Значение интеграла: $ 0.188246831$
	\item Оценка погрешности: $3.13 \times 10^{-7}$ (что меньше $\varepsilon$)
\end{itemize}

	\begin{figure}[H] 
	\center
	\includegraphics[width=0.8\textwidth]{Задание2Результат.png}
	\caption{Результат работы программы} 
\end{figure}

\subsection{Проверка точности}
Для верификации результатов проведена проверка с $n=16$:
\begin{itemize}
	\item Значение интеграла: $0.18824652$
	\item Разница с предыдущим: $3.09 \times 10^{-7}$
\end{itemize}

\subsection{Визуализация}
	\begin{figure}[H] 
	\center
	\includegraphics[width=0.8\textwidth]{Задание2График.png}
	\caption{Исходный график программы} 
\end{figure}
График показывает поведение интегрируемой функции. Особенности:
\begin{itemize}
	\item Функция монотонно убывает на заданном интервале
	\item Имеет гладкий характер, что благоприятно для метода Симпсона
\end{itemize}
Результат программы сошелся с проверкой в стороннем ресурсе.
	\begin{figure}[H] 
	\center
	\includegraphics[width=0.8\textwidth]{Задание2Проверка.png}
	\caption{Проверка работы программы} 
\end{figure}


\subsection{Выводы}
Метод Симпсона показал высокую эффективность:
\begin{itemize}
	\item Достигнута требуемая точность $10^{-4}$ при $n=8$
	\item Оценка погрешности совпадает с реальной разницей при удвоении $n$
	\item Метод требует в 2 раза меньше разбиений по сравнению с методом трапеций для аналогичной точности
\end{itemize}

\subsection*{Реализация на Python}
\lstinputlisting[language=Python, caption={Python код программы}]{Laba4_2.py}
\section{Численное интегрирование методом Гаусса}

\subsection{Постановка задачи}
Требуется вычислить определённый интеграл функции 
\[ f(x) = \frac{x}{\sqrt{x^2 + 2.5}} \]
на отрезке $[1.4, 2.6]$ методом квадратур Гаусса.

\subsection{Теоретическая основа}
Метод Гаусса вычисляет интеграл по формуле:
\[
\int_a^b f(x)dx \approx \frac{b-a}{2}\sum_{i=1}^n w_i f\left(\frac{b-a}{2}x_i + \frac{a+b}{2}\right)
\]
где:
\begin{itemize}
	\item $x_i$ --- узлы квадратуры на интервале $[-1, 1]$
	\item $w_i$ --- весовые коэффициенты
	\item $n$ --- количество узлов (4 или 7 в данной реализации)
\end{itemize}

Для $n=4$ и $n=7$ используются стандартные табличные значения узлов и весов.

\subsection{Реализация}

\subsubsection{Узлы и веса}
\begin{center}
	\begin{tabular}{cc|cc}
		\toprule
		\multicolumn{2}{c|}{4 узла} & \multicolumn{2}{c}{7 узлов} \\
		Узел $x_i$ & Вес $w_i$ & Узел $x_i$ & Вес $w_i$ \\
		\midrule
		-0.86114 & 0.34785 & -0.949108 & 0.129485 \\
		-0.33998 & 0.65215 & -0.741531 & 0.279705 \\
		0.33998 & 0.65215 & -0.405845 & 0.381830 \\
		0.86114 & 0.34785 & 0.000000 & 0.417959 \\
		& & 0.405845 & 0.381830 \\
		& & 0.741531 & 0.279705 \\
		& & 0.949108 & 0.129485 \\
		\bottomrule
	\end{tabular}
\end{center}

\subsubsection{Вычисление интеграла}
Преобразование координат:
\[
\xi \mapsto \frac{b-a}{2}x + \frac{a+b}{2}
\]


\subsection{Результаты}
\begin{figure}[H]
	\center
	\includegraphics[width=0.8\textwidth]{Задание3Результат.png}
	\caption{Результат работы программы}
\end{figure}
Проверка работы программы в сторонних ресурсах. Ответ сошелся.
\begin{figure}[H]
	\center
	\includegraphics[width=0.8\textwidth]{Задание3Проверка.png}
	\caption{Проверка работы программы}
\end{figure}


\begin{itemize}
	\item При $n=4$: $0.931156$ (6 значащих цифр)
	\item При $n=7$: $0.931153$ (6 значащих цифр)
\end{itemize}

Разница между результатами составляет $1\times10^{-7}$, что демонстрирует быструю сходимость метода.

\subsection{Визуализация}
\begin{figure}[H]
	\center
	\includegraphics[width=0.8\textwidth]{Задание3График.png}
	\caption{График исходной функции}
\end{figure}

Анализ функции:
\begin{itemize}
	\item Функция монотонно возрастает на заданном интервале
	\item Имеет гладкий характер без особенностей
	\item Хорошо аппроксимируется полиномами, что объясняет эффективность метода Гаусса
\end{itemize}

\subsection{Выводы}
Метод Гаусса показал высокую точность:
\begin{itemize}
	\item Уже при 4 узлах достигается 6 значащих цифр
	\item Увеличение числа узлов до 7 даёт незначительное уточнение
	\item Метод эффективен для гладких функций без особенностей
\end{itemize}
\subsection*{Реализация на Python}
\lstinputlisting[language=Python, caption={Python код программы}]{Laba4_3.py}
\section{Численное решение ОДУ методом Рунге-Кутты 2-го порядка}

\subsection{Постановка задачи}
Требуется решить дифференциальное уравнение:
\[ y' = x^2 - y^2 \]
с начальным условием $y(0) = 0$ на интервале $x \in [0, 1]$ методом Рунге-Кутты 2-го порядка.

\subsection{Теоретическая основа}
Метод Рунге-Кутты 2-го порядка (модифицированный метод Эйлера) имеет вид:
\[
y_{i+1} = y_i + \frac{h}{2}(k_1 + k_2)
\]
где:
\begin{align*}
	k_1 &= f(x_i, y_i) \\
	k_2 &= f(x_i + h, y_i + hk_1)
\end{align*}

Локальная погрешность метода $O(h^3)$, глобальная $O(h^2)$.

\subsection{Реализация}

\subsubsection{Алгоритм}
\begin{enumerate}
	\item Задаём начальные условия $x_0$, $y_0$ и шаг $h$
	\item На каждом шаге вычисляем:
	\begin{itemize}
		\item Коэффициент $k_1 = h \cdot f(x_i, y_i)$
		\item Коэффициент $k_2 = h \cdot f(x_i + h, y_i + k_1)$
		\item Новое значение $y_{i+1} = y_i + \frac{1}{2}(k_1 + k_2)$
	\end{itemize}
	\item Повторяем до достижения конечной точки $x_{end}$
\end{enumerate}

\subsection{Результаты}

\subsubsection{Табличные значения}
\begin{center}
	\begin{tabular}{cc|cc}
		\toprule
		\multicolumn{2}{c|}{$h=0.1$} & \multicolumn{2}{c}{$h=0.05$} \\
		$x$ & $y(x)$ & $x$ & $y(x)$ \\
		\midrule
		0.0 & 0.000000 & 0.0 & 0.000000 \\
		0.1 & 0.000333 & 0.1 & 0.000333 \\
		0.2 & 0.002667 & 0.2 & 0.002667 \\
		0.3 & 0.009000 & 0.3 & 0.009000 \\
		0.4 & 0.021330 & 0.4 & 0.021331 \\
		0.5 & 0.041663 & 0.5 & 0.041665 \\
		0.6 & 0.072000 & 0.6 & 0.072005 \\
		0.7 & 0.114330 & 0.7 & 0.114341 \\
		0.8 & 0.170663 & 0.8 & 0.170684 \\
		0.9 & 0.243000 & 0.9 & 0.243037 \\
		1.0 & 0.333330 & 1.0 & 0.333392 \\
		\bottomrule
	\end{tabular}
\end{center}


\subsection{Визуализация}
\begin{figure}[H]
	\centering
	\includegraphics[width=0.8\textwidth]{Задание4График.png}
	\caption{Сравнение решений для разных шагов}
\end{figure}

На графике видно:
\begin{itemize}
	\item Решения для $h=0.1$ и $h=0.05$ практически совпадают
	\item Меньший шаг даёт более гладкую кривую
	\item Метод демонстрирует хорошую сходимость
\end{itemize}

\subsection*{Реализация на Python}
\lstinputlisting[language=Python, caption={Python код программы}]{Laba4_4.py}



\section{Вывод}
В ходе данной лабораторной работы, я смог ознакомлся с 3 методами интегрирования. А именно: метод трапеций, метод Cимпсона, метод Гаусса.Также было решено диффиринциальное уравнение методом Рунге-Кутты 2-го порядка Эти методы были реализованы в программе с использованием высокого языка программирования (Python). Также вычесленные корни были проверены, они сошлись с минимальной погрешностью.
\end{document}


