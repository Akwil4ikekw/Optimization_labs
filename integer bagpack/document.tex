\documentclass[a4paper,12pt]{extarticle}
\usepackage[T1, T2A]{fontenc}
\usepackage{textcomp} %Спецсимволы
\usepackage{graphicx} %Графика
\usepackage{ragged2e} %Для использования {justify} - по ширине
\usepackage[english, russian]{babel}
\usepackage{setspace}
\usepackage{titlesec}
\usepackage[colorlinks=true,urlcolor=blue]{hyperref}  
\usepackage{float} % Для принудительного размещения float-объектов

\usepackage[a4paper, papersize={210mm, 297mm}, text={210mm, 297mm},
left=1.5cm, top=2cm, right=1.5cm, bottom=2cm]{geometry}

\usepackage{fvextra}
\DefineVerbatimEnvironment{code}{Verbatim}{breaklines=true}

\usepackage{etoolbox}
\AtBeginEnvironment{code}{\linespread{1.0}\selectfont}

\usepackage[font={normalfont}]{caption}
\captionsetup[figure]{name=Рисунок, labelsep=endash}

\setlength{\parskip}{0.5cm} %Отступ после абзаца
\linespread{1.5} %Междустрочнный интервал

\usepackage{enumitem}
\setlist[enumerate]{itemindent=\parskip}
\setlist[itemize]{itemindent=\parskip}

\setlength{\parindent}{1.5cm}

\titleformat{\section}
{\normalfont\bfseries}
{\thesection}{0.5em}{}

\titleformat{\subsection}
{\normalfont\bfseries\normalsize}
{\thesubsection}{0.5em}{}

\titlespacing{\section}{\parindent}{0pt}{0pt}
\titlespacing{\subsection}{\parindent}{0pt}{0pt}

\begin{document}
	
	\pagestyle{empty}
	
	\begin{center}
		
		\begin{onehalfspace}
			
			МИНИСТЕРСТВО НАУКИ И ВЫСШЕГО ОБРАЗОВАНИЯ \\
			РОССИЙСКОЙ ФЕДЕРАЦИИ \\
			
			\vspace{2mm}
			
			ФЕДЕРАЛЬНОЕ ГОСУДАРСТВЕННОЕ БЮДЖЕТНОЕ ОБРАЗОВАТЕЛЬНОЕ\\
			УЧРЕЖДЕНИЕ ВЫСШЕГО ОБРАЗОВАНИЯ\\
			«ВЯТСКИЙ ГОСУДАРСТВЕННЫЙ УНИВЕРСИТЕТ»\\
			
			\vspace{2mm}
			
			Институт математики и информационных систем\\
			\vspace{2mm}
			
			Факультет автоматики и вычислительной техники \\
			\vspace{2mm}
			
			Кафедра электронных вычислительных машин \\
			
		\end{onehalfspace}
		
	\end{center}
	
	\begin{flushright}
		\begin{small}
			Дата сдачи на проверку: \\
			«\underline{\hspace{5mm}}»
			\underline{\hspace{21mm}}
			2025 г. \\
			Проверка: \rule[0ex]{25mm}{0pt} \\
			«\underline{\hspace{5mm}}»
			\underline{\hspace{21mm}}
			2025 г. 
		\end{small}
	\end{flushright}
	
	\begin{center}
		
		
		\textbf{ЛАБОРАТОРНАЯ РАБОТА}
		
		Отчет по лабораторной работе №8 \\
		по дисциплине
		
		Вычислительная математика и методы оптимизаций
		
		\vspace{20mm}
		
		\begin{tabular}{ll}
			Разработал студент гр. ИВТб-2302-05-00 & \rule[0ex]{30mm}{0.1pt}~/Соловьев А.С./~ \\
			&  \rule[0ex]{8mm}{0pt}\scriptsize{(Подпись)} \\
			
			Проверил старший преподаватель &\rule[0ex]{30mm}{0.1pt}~/Коржавина А.С./~ \\
			&  \rule[0ex]{8mm}{0pt}\scriptsize{(Подпись)} \\
			
			Работа защищена & \hfill «\rule[0ex]{5mm}{0pt}»\rule[0ex]{24mm}{0.1pt}~2025г.~ 
			
		\end{tabular}
		
	\end{center}
	
	
	\vspace{8mm}
	\begin{center}
		Киров 2025
	\end{center}
	
	\newpage
	
	\pagestyle{plain}
	\setcounter{page}{2}
	\textbf{Дано}
	\begin{itemize}
		\item $max\_space$ --- максимальное доступное пространство (вес).
		\item $w = (w_1, w_2, \ldots, w_n)$ --- веса предметов.
		\item $price\_item = (p_1, p_2, \ldots, p_n)$ --- стоимости предметов.
		\item $counts = (c_1, c_2, \ldots, c_n)$ --- максимальное количество доступных копий каждого предмета.
	\end{itemize}
	
	\bigskip
	
	\textbf{Задача:}
	
	Найти такой набор предметов (с учетом ограничений на количество каждого предмета), чтобы суммарный вес не превышал $max\_space$, а суммарная стоимость была максимальной.
	
	\bigskip
	
	\textbf{Алгоритм (динамическое программирование):}
	
	Определим массив $d[i][j]$ --- максимальную стоимость, которую можно получить, используя первые $i$ предметов и не превышая вес $j$.
	
	\[
	d[0][j] = 0, \quad \forall j = 0, \ldots, max\_space
	\]
	
	Для каждого предмета $i = 1, \ldots, n$ и каждого возможного веса $j = 0, \ldots, max\_space$ вычисляем:
	
	\[
	d[i][j] = \max_{k=0}^{\min(c_i, \lfloor j / w_i \rfloor)} \left( d[i-1][j - k \cdot w_i] + k \cdot p_i \right)
	\]
	
	где $k$ --- количество взятых копий $i$-го предмета.
	
	\bigskip
	
	Для восстановления решения используется вспомогательный массив $used[i][j]$, который хранит количество предметов $i$ в оптимальном наборе при весе $j$.
	
	\bigskip
	
	\textbf{После заполнения таблицы:}
	
	\begin{itemize}
		\item Максимальная стоимость: $d[n][max\_space]$.
		\item Для восстановления выбранных предметов идём от $d[n][max\_space]$ назад, вычитая веса выбранных копий.
	\end{itemize}
	
	\begin{figure}[H] 
		\center
		\includegraphics [width=0.8\linewidth]{Program.png}
		\caption{Конечный результат работы программа }
	\end{figure} 
	
	
	
	\begin{center}
		\href{https://github.com/Akwil4ikekw/Optimization_labs.git}{Ссылка на репозиторий GitHub}
	\end{center}
	
	
	
	\section{Вывод}
	
	В ходе данной лабораторной работы была провелена деятельность по созданию алгоритма для реализации целочисленного рюкзака.
	
\end{document}