\documentclass[a4paper,14pt]{extarticle}
\usepackage[T1, T2A]{fontenc}
\usepackage{textcomp}

\usepackage[utf8]{inputenc}
\usepackage{listings}
\usepackage{xcolor}  % Для цветов в коде

\lstset{
	language=Python,
	basicstyle=\ttfamily\footnotesize,
	numbers=left,
	numberstyle=\tiny\color{gray},
	stepnumber=1,
	numbersep=5pt,
	backgroundcolor=\color{white},
	showspaces=false,
	showstringspaces=false,
	showtabs=false,
	frame=single,
	rulecolor=\color{black},
	tabsize=2,
	captionpos=b,
	breaklines=true,
	breakatwhitespace=false,
	escapeinside={\%*}{*)},
	keywordstyle=\color{blue},
	commentstyle=\color{green},
	stringstyle=\color{red},
	literate={а}{{\selectfont\char224}}1
	{б}{{\selectfont\char225}}1
	{в}{{\selectfont\char226}}1
	{г}{{\selectfont\char227}}1
	{д}{{\selectfont\char228}}1
	{е}{{\selectfont\char229}}1
	{ё}{{\selectfont\char184}}1
	{ж}{{\selectfont\char230}}1
	{з}{{\selectfont\char231}}1
	{и}{{\selectfont\char232}}1
	{й}{{\selectfont\char233}}1
	{к}{{\selectfont\char234}}1
	{л}{{\selectfont\char235}}1
	{м}{{\selectfont\char236}}1
	{н}{{\selectfont\char237}}1
	{о}{{\selectfont\char238}}1
	{п}{{\selectfont\char239}}1
	{р}{{\selectfont\char240}}1
	{с}{{\selectfont\char241}}1
	{т}{{\selectfont\char242}}1
	{у}{{\selectfont\char243}}1
	{ф}{{\selectfont\char244}}1
	{х}{{\selectfont\char245}}1
	{ц}{{\selectfont\char246}}1
	{ч}{{\selectfont\char247}}1
	{ш}{{\selectfont\char248}}1
	{щ}{{\selectfont\char249}}1
	{ъ}{{\selectfont\char250}}1
	{ы}{{\selectfont\char251}}1
	{ь}{{\selectfont\char252}}1
	{э}{{\selectfont\char253}}1
	{ю}{{\selectfont\char254}}1
	{я}{{\selectfont\char255}}1
	{А}{{\selectfont\char192}}1
	{Б}{{\selectfont\char193}}1
	{В}{{\selectfont\char194}}1
	{Г}{{\selectfont\char195}}1
	{Д}{{\selectfont\char196}}1
	{Е}{{\selectfont\char197}}1
	{Ё}{{\selectfont\char168}}1
	{Ж}{{\selectfont\char198}}1
	{З}{{\selectfont\char199}}1
	{И}{{\selectfont\char200}}1
	{Й}{{\selectfont\char201}}1
	{К}{{\selectfont\char202}}1
	{Л}{{\selectfont\char203}}1
	{М}{{\selectfont\char204}}1
	{Н}{{\selectfont\char205}}1
	{О}{{\selectfont\char206}}1
	{П}{{\selectfont\char207}}1
	{Р}{{\selectfont\char208}}1
	{С}{{\selectfont\char209}}1
	{Т}{{\selectfont\char210}}1
	{У}{{\selectfont\char211}}1
	{Ф}{{\selectfont\char212}}1
	{Х}{{\selectfont\char213}}1
	{Ц}{{\selectfont\char214}}1
	{Ч}{{\selectfont\char215}}1
	{Ш}{{\selectfont\char216}}1
	{Щ}{{\selectfont\char217}}1
	{Ъ}{{\selectfont\char218}}1
	{Ы}{{\selectfont\char219}}1
	{Ь}{{\selectfont\char220}}1
	{Э}{{\selectfont\char221}}1
	{Ю}{{\selectfont\char222}}1
	{Я}{{\selectfont\char223}}1
}
\usepackage{amsmath}
\usepackage[normalem]{ulem} 
\usepackage{tabularray}
\usepackage{graphicx}
\usepackage{diagbox} 
\usepackage{multirow}
\usepackage{float}
\usepackage{wrapfig}
\usepackage{ragged2e}
\usepackage[english, russian]{babel}
\usepackage{setspace}
\setlength{\parskip}{0.5cm}
\linespread{1.5}
\setlength{\parindent}{0.5cm}
\usepackage[a4paper, papersize={210mm, 297mm}, text={210mm, 297mm},
left=2cm, top=2cm, right=1.5cm, bottom=2cm]{geometry}
\begin{document} 
	\begin{center}
		\normalsize{МИНИСТЕРСТВО НАУКИ И ВЫСШЕГО ОБРАЗОВАНИЯ
			РОССИЙСКОЙ ФЕДЕРАЦИИ}\\
		\normalsize{ФЕДЕРАЛЬНОЕ ГОСУДАРСТВЕННОЕ БЮДЖЕТНОЕ ОБРАЗОВАТЕЛЬНОЕ
			УЧРЕЖДЕНИЕ ВЫСШЕГО ОБРАЗОВАНИЯ\\
			«ВЯТСКИЙ ГОСУДАРСТВЕННЫЙ УНИВЕРСИТЕТ»}\\ 
		\normalsize{Институт математики и информационных систем}\\
		\normalsize{Факультет автоматики и вычислительной техники}\\
		\normalsize{Кафедра электронных вычислительных машин}\\
	\end{center}
	
	\begin{flushright}
		\small{Дата сдачи на проверку:}\\
		\small{«}\underline{\hspace{0.5cm}}\small{»}\underline{\hspace{2cm}} \small{2025 г.}\\
		\small{Проверено:} \rule[0ex]{2.3cm}{0pt}\\
		\small{«}\underline{\hspace{0.5cm}}\small{»}\underline{\hspace{2cm}}\small{2025 г.} 
	\end{flushright}
	\begin{center}
		\normalsize{Вариант 18}\\
		\normalsize{Отчет по лабораторной работе № 2 }\\ по дисциплине \\ 
		\normalsize{«Вычислительная математика»}\\
		
		
	\end{center}
	
	
	\begin{flushleft}
		\hfill \break
		\hfill \break
		\normalsize{ Разработал студент гр. ИВТб 2302-05-00}\rule[0ex]{0.1cm}{0pt} \underline{\hspace{4cm}}  \normalsize{/Соловьев А.С./} \\
		\rule[0ex]{9.6cm}{0pt} \tiny{(Подпись)}\\
		\normalsize{ Проверил заведующий кафедры ЭВМ}  \rule[0ex]{0.1cm}{0pt} \underline{\hspace{4cm}} \normalsize{/Старостин П.А./} \\ 
		\rule[0ex]{9.6cm}{0pt} \tiny{(Подпись)}\\
		\normalsize{ Работа защищена} \rule[0ex]{8.5cm}{0pt} \small{«}\underline{\hspace{0.5cm}}\small{»}\underline{\hspace{2.6cm}}\small{2025 г.}
		\hfill \break
		\hfill \break
	\end{flushleft}
	\begin{center} Киров 2025 \end{center}
	\thispagestyle{empty} % выключаем отображение номера для этой страницы
	
	\newpage
	
	\section{Задания}
	\textbf{Задание 1}

		Решить систему линейных уравнений 4-го порядка методом Гаусса с точностью е=0,001.
		Уравнение системы:
		 \begin{equation*}
		 \begin{cases}
			       0,17*x1-0,13*x2-0,11*x3-0,12*x4=0,22        \\
		1,00*x1-1,00*x2-0,13*x3+0,13*x4=0,11  \\
		0,35*x1+0,33*x2+0,12*x3+0,13*x4=0,12       \\ 
	0,13*x1+0,11*x2-0,13*x3-0,11*x4=1,00\\
	
		\end{cases}
	\end{equation*}

\textbf{Задание 2}\\
Решить систему линейных уравнений 4-го порядка с точностью e=0,0001: методом простой итерации.
Уравнение системы:
 \begin{equation*}
	\begin{cases}
    x1=0,23*x1-0,04*x2+0,21*x3-0,18*x4+1,24      \\
		x2=0,45*x1-0,23*x2+0,06*x3-0,88       \\
	x3=0,26*x1+0,34*x2-0,11*x3+0,62    \\ 
	x4=0,05*x1-0,26*x2+0,34*x3-0,12*x4-1,17\\
	\end{cases}
\end{equation*}

\newpage
\textbf{Задание 3}\\
Решить систему линейных уравнений 3-го порядка методом обратной матрицы с точностью до e=0.001
 \begin{equation*}
	\begin{cases}
		  4*x1+8*x2+7*x3=9       \\
		x1+2*x2+2*x3=2       \\
		2*x1+3*x2+x3=9   \\ 
		
	\end{cases}
\end{equation*}

             \textbf{Задание 4} \\
Решить систему нелийнейных уравнений 2-го порядка метод Ньютона с точность e=0.001: 
Уравнение системы:
              
\begin{equation*}
	\begin{cases}
		30*x^2+7*y^2-1=0            \\
		\sin(4*x-0,5*y)+5*x=0      \\
	\end{cases}
	
\end{equation*}


	
	
	\section{Теорическая часть.Описание методов решения уравнений}
	
	
	

	\begin{enumerate}
		\item \textbf{Метод Гаусса}\\
		Метод Гаусса состоит из 2 действий: прямого и обратного хода Гаусса. Прямой ход - привведние матрицы к ступенчатому виду. Обратный ход - последовательное нахождение корней системы. 


		Прямой метод Гаусса заключается в приведении матрицы $A$ к верхнетреугольному виду с помощью элементарных преобразований строк(сложение строк, домножение на константу).
		
		Для каждого шага $k$ (от $1$ до $n-1$):
		
		1. Выбираем ведущий элемент $a_{kk}$.
		2. Обнуляем элементы ниже диагонали в $k$-м столбце:
		\[
		a_{ik} = a_{ik} - \frac{a_{ik}}{a_{kk}} a_{kk}, \quad \forall i > k.
		\]
		3. Обновляем правую часть:
		\[
		b_i = b_i - \frac{a_{ik}}{a_{kk}} b_k, \quad \forall i > k.
		\]
		
		После $n-1$ шагов матрица принимает **верхнетреугольный вид**:
		
		\[
		U =
		\begin{bmatrix}
			u_{11} & u_{12} & \dots & u_{1n} \\
			0 & u_{22} & \dots & u_{2n} \\
			0 & 0 & \ddots & u_{(n-1)n} \\
			0 & 0 & 0 & u_{nn}
		\end{bmatrix}.
		\]
		\\
		\item\textbf{Обратный ход}
		
		Теперь решаем **треугольную систему** сверху вниз:
		
		\[
		u_{nn}x_n = b_n \Rightarrow x_n = \frac{b_n}{u_{nn}}.
		\]
		
		Затем находим остальные неизвестные по формуле:
		
		\[
		x_i = \frac{b_i - \sum_{j=i+1}^{n} u_{ij}x_j}{u_{ii}}, \quad i = n-1, n-2, \dots, 1.
		\]
		\end{enumerate}
		
		\section{Метод простых итераций}
		Рассмотрим систему линейных уравнений:
		\begin{enumerate}
		\begin{equation}
			A \mathbf{x} = \mathbf{b},
		\end{equation}
		где:
		\begin{itemize}
			\item \( A \) — квадратная матрица коэффициентов размерности \( n \times n \);
			\item \( \mathbf{x} = (x_1, x_2, \dots, x_n)^T \) — вектор неизвестных;
			\item \( \mathbf{b} = (b_1, b_2, \dots, b_n)^T \) — вектор свободных членов.
		\end{itemize}
		
		Метод простых итераций (метод последовательных приближений) заключается в переходе к эквивалентному виду:
		
		\begin{equation}
			\mathbf{x} = P \mathbf{x} + \mathbf{g},
		\end{equation}
		где \( P \) — итерационная матрица, а \( \mathbf{g} \) — вектор.
		
		\item\textbf{Итерационный процесс}\\
		Рекуррентная формула метода имеет вид:
		\begin{equation}
			\mathbf{x}^{(k+1)} = P \mathbf{x}^{(k)} + \mathbf{g}, \quad k = 0,1,2, \dots
		\end{equation}
		
		Процесс продолжается до выполнения условия остановки:
		\begin{equation}
			\|\mathbf{x}^{(k+1)} - \mathbf{x}^{(k)}\| \leq \varepsilon,
		\end{equation}
		где \( \varepsilon \) — заданная точность.
		
		\item\textbf{Алгоритм метода}
		1. Выбираем начальное приближение \( \mathbf{x}^{(0)} \).
		2. Вычисляем следующее приближение:
		\[
		\mathbf{x}^{(k+1)} = P \mathbf{x}^{(k)} + \mathbf{g}.
		\]
		3. Проверяем условие сходимости:
		\[
		\|\mathbf{x}^{(k+1)} - \mathbf{x}^{(k)}\| \leq \varepsilon.
		\]
		Если оно выполнено, процесс завершается.
		4. Если условие не выполнено, увеличиваем \( k \) и повторяем шаги 2-3.
		\end{enumerate}
		
		
\section{Метод LU разложения с перестановками}
LU-разложение представляет собой представление квадратной матрицы \(A\) в виде произведения трех матриц:
\[
A = P L U,
\]
где \(P\) — матрица перестановок, \(L\) — нижняя треугольная матрица, а \(U\) — верхняя треугольная матрица. Это разложение позволяет решить систему линейных уравнений и вычислять определители и обратные матрицы.

\\
Метод LU-разложения с перестановками используется для приведения матрицы \(A\) к верхней и нижней треугольной форме с учетом возможных проблем с числовой устойчивостью. В отличие от стандартного метода LU-разложения, при котором матрица \(A\) может не быть разложена без ошибок, в методе с перестановками строки матрицы переставляются так, чтобы в процессе преобразований не возникали деления на ноль или слишком малые числа.

Процесс разложения можно описать следующим образом:

\begin{enumerate}
	\item Перестановка строк: Для обеспечения стабильности выбираются такие строки, которые минимизируют числовые погрешности при делении. Это делается с помощью матрицы перестановок \(P\), которая переставляет строки исходной матрицы \(A\).
	\item Приведение к верхней треугольной форме: Затем с использованием метода Гаусса, но с учетом перестановок строк, матрица \(A\) приводится к верхней треугольной форме \(U\).
	\item Получение нижней треугольной матрицы: После приведения к верхней треугольной матрице элементы, которые не являются главной диагональю, записываются в нижнюю треугольную матрицу \(L\). При этом на главной диагонали матрицы \(L\) стоят единицы.
\end{enumerate}

Таким образом, результатом разложения является три матрицы:
\[
A = P L U,
\]
где:
\[
P = \text{матрица перестановок}, \quad L = \text{нижняя треугольная матрица}, \quad U = \text{верхняя треугольная матрица}.
\]

\textbf{\large{Решение системы линейных уравнений}}
Рассмотрим систему линейных уравнений:
\[
A \mathbf{x} = \mathbf{b},
\]
где \(A\) — матрица коэффициентов, \(\mathbf{x}\) — вектор неизвестных, \(\mathbf{b}\) — вектор правых частей.

Для решения этой системы с использованием LU-разложения с перестановками, сначала необходимо разложить матрицу \(A\) на \(P\), \(L\) и \(U\), так что:
\[
A = P L U.
\]
Затем можно записать систему как:
\[
P L U \mathbf{x} = \mathbf{b}.
\]
\[
L y = P \mathbf{b} 
\]
Находим значения для каждого $y_i$
по формуле
 \[
 y_i = b_i - \sum_{k=0}^{i-1} L_i_k y_k}
 \]
 Найдем знаечние для каждого $x_i$ по формуле
 \[
  x_i = y_i - \sum_{k=0}^{i-1} \frac{U_i_k  x_k}{  U_i_i}

 \]
Мы получили решение для данной системы.

\subsection{Вычисление обратной матрицы с использованием LU-разложения}

Обратная матрица \(A^{-1}\) может быть вычислена с использованием LU-разложения с перестановками. Поскольку \(A = P L U\), то для вычисления \(A^{-1}\) необходимо решить систему линейных уравнений для каждого столбца единичной матрицы.

Процесс нахождения обратной матрицы включает следующие шаги:

  Для каждого столбца единичной матрицы \(I\) решается система:
\[
A \mathbf{x}_i = \mathbf{e}_i,
\]
где \(\mathbf{e}_i\) — \(i\)-й столбец единичной матрицы.

 . После разложения \(A = P L U\), эта система может быть записана как:
\[
P L U \mathbf{x}_i = \mathbf{e}_i.
\]



После того, как мы получим все столбцы \(\mathbf{x}_i\), они составляют столбцы обратной матрицы \(A^{-1}\).

Таким образом, метод LU-разложения с перестановками не только позволяет решать системы линейных уравнений, но и вычислять обратную матрицу. При этом процесс нахождения обратной матрицы сводится к решению системы линейных уравнений для каждого столбца единичной матрицы, что делает метод эффективным и численно стабильным.



\section{Алгоритм решения СНАУ методом Ньютона}
Задача состоит в нахождении корней системы нелинейных алгебраических уравнений:

\[
F(\mathbf{x}) = \begin{pmatrix} f_1(\mathbf{x}) \\ f_2(\mathbf{x}) \\ \vdots \\ f_n(\mathbf{x}) \end{pmatrix} = 0
\]

где \(\mathbf{x} = \begin{pmatrix} x_1 \\ x_2 \\ \vdots \\ x_n \end{pmatrix}\) — вектор переменных, а \(F(\mathbf{x})\) — вектор функций.

Метод Ньютона для системы нелинейных уравнений представляет собой итерационный процесс, который записывается следующим образом:

\[
\mathbf{x}_{k+1} = \mathbf{x}_k - J^{-1}(\mathbf{x}_k) F(\mathbf{x}_k)
\]

где \(J(\mathbf{x}_k)\) — якобиан системы, то есть матрица частных производных:

\[
J(\mathbf{x}) = \begin{pmatrix}
	\frac{\partial f_1}{\partial x_1} & \frac{\partial f_1}{\partial x_2} & \cdots & \frac{\partial f_1}{\partial x_n} \\
	\frac{\partial f_2}{\partial x_1} & \frac{\partial f_2}{\partial x_2} & \cdots & \frac{\partial f_2}{\partial x_n} \\
	\vdots & \vdots & \ddots & \vdots \\
	\frac{\partial f_n}{\partial x_1} & \frac{\partial f_n}{\partial x_2} & \cdots & \frac{\partial f_n}{\partial x_n}
\end{pmatrix}
\]

Алгоритм решения:

\begin{enumerate}
	\item Выберите начальное приближение \(\mathbf{x}_0\).
	\item Для каждого шага \(k\) выполните следующие действия:
	\begin{enumerate}
		\item Вычислите \(F(\mathbf{x}_k)\) и \(J(\mathbf{x}_k)\).
		\item Решите линейную систему:
		\[
		\Delta \mathbf{x}_k = -J^{-1}(\mathbf{x}_k) F(\mathbf{x}_k)
		\]
		\item Обновите значение вектора \(\mathbf{x}_k\):
		\[
		\mathbf{x}_{k+1} = \mathbf{x}_k + \Delta \mathbf{x}_k
		\]
	\end{enumerate}
	\item Повторяйте шаги до тех пор, пока не будет достигнута желаемая точность:
	\[
	\| F(\mathbf{x}_{k+1}) \| < \epsilon
	\]
	где \(\epsilon\) — заданная точность.
\end{enumerate}




	\section{Практическая часть}
	\textbf{Задание 1}

Решить систему линейных уравнений 4-го порядка методом Гаусса с точностью е=0,001.
Уравнение системы:
 \begin{equation*}
	\begin{cases}
		0,17*x1-0,13*x2-0,11*x3-0,12*x4=0,22        \\
		1,00*x1-1,00*x2-0,13*x3+0,13*x4=0,11  \\
		0,35*x1+0,33*x2+0,12*x3+0,13*x4=0,12       \\ 
		0,13*x1+0,11*x2-0,13*x3-0,11*x4=1,00\\
		
	\end{cases}
\end{equation*}

Эту систему можно записать в матричной форме:

\[
A \cdot \mathbf{x} = \mathbf{b}
\]

где \(A\) — матрица коэффициентов, \(\mathbf{x} = \begin{pmatrix} x_1 \\ x_2 \\ x_3 \\ x_4 \end{pmatrix}\) — вектор переменных, а \(\mathbf{b} = \begin{pmatrix} 0.22 \\ 0.11 \\ 0.12 \\ 1.00 \end{pmatrix}\) — вектор правых частей уравнений.

Метод Гаусса состоит из двух этапов:
\begin{enumerate}
	\item \textbf{Прямой ход}: Приведение матрицы коэффициентов \(A\) к верхнетреугольному виду с помощью элементарных преобразований строк. В ходе этого этапа мы находим коэффициенты, которые позволяют обнулить элементы ниже главной диагонали.
	\item \textbf{Обратный ход}: После приведения матрицы к верхнетреугольному виду, начинаем вычисление переменных системы, начиная с последней строки. Для каждой строки вычисляется переменная как:
	
	\[
	x_i = \frac{b_i - \sum_{j=i+1}^{n} a_{ij} x_j}{a_{ii}}
	\]
\end{enumerate}

\textbf{\large{Прямой ход}}
На первом шаге мы применяем метод выбора ведущего элемента, чтобы минимизировать погрешности вычислений. В ходе этого шага мы находим максимальные элементы по столбцам и меняем строки, если это необходимо.

Затем, используя элементарные преобразования строк, мы приводим систему к верхнетреугольному виду. Для каждой строки \(i\) вычисляется коэффициент:

\[
k_{ij} = \frac{a_{ij}}{a_{ii}}
\]

и вычитаем \(k_{ij}\) умноженное на строку \(i\) из строки \(j\), чтобы обнулить элементы ниже ведущего.

\textbf{\large{Обратный ход}}
После приведения матрицы к верхнетреугольному виду, мы начинаем обратный ход. Сначала решаем для последней переменной, затем для остальных, двигаясь вверх по строкам.

Решение для переменной \(x_4\) на основе последней строки:

\[
x_4 = \frac{b_4 - (a_{43} x_3 + a_{44} x_4)}{a_{44}}
\]

Затем, по аналогии, решаем для остальных переменных \(x_3\), \(x_2\), и \(x_1\).

\textbf{\large{Решение}}
После применения метода Гаусса система уравнений была решена. Корни, полученные методом Гаусса:

\[
x_1 = -0.044\pm 0.001, \quad x_2 = 2.068\pm 0.001, \quad x_3 = -11.080\pm 0.001, \quad x_4 = 6.019\pm 0.001
\]
	\begin{figure}[H] 
	\center
	\includegraphics [width=0.35\linewidth]{Корни методом Гаусса.png}
	\caption{Решение методом Гаусса}
\end{figure} 
Проверим найденные корни в Maxima
	\begin{figure}[H] 
	\center
	\includegraphics [width=0.35\linewidth]{ГауссМаксима.png}
	\caption{Решение СЛАУ в Максима}
\end{figure} 
Найденные корни совпадают с корнями, найденными в программе python.
\textbf{Листинг программы: }
\lstinputlisting[language=Python, caption={Python код программы}]{Gauss.py}


\textbf{Задание 2}\\
Решить систему линейных уравнений 4-го порядка с точностью e=0,0001: методом простой итерации.
Уравнение системы:
\begin{equation*}
	\begin{cases}
		x1=0,23*x1-0,04*x2+0,21*x3-0,18*x4+1,24      \\
		x2=0,45*x1-0,23*x2+0,06*x3-0,88       \\
		x3=0,26*x1+0,34*x2-0,11*x3+0,62    \\ 
		x4=0,05*x1-0,26*x2+0,34*x3-0,12*x4-1,17\\
	\end{cases}
\end{equation*}

Метод простых итераций используется для численного решения данной системы. В методе простых итераций на каждом шаге обновляются значения переменных, пока разность между старым и новым значением не станет меньше заданной погрешности.

\section*{Шаги решения}

\subsection*{1. Представление системы уравнений в виде матрицы и вектора свободных членов:}

Матрица коэффициентов системы:
\[
P = \begin{bmatrix}
	0.23 & 0.04 & 0.21 & -0.18 \\
	0.45 & 0.23 & 0.06 & 0 \\
	0.26 & 0.34 & -0.11 & 0 \\
	0.05 & -0.26 & 0.34 & -0.11
\end{bmatrix}
\]
Вектор свободных членов:
\[
g = \begin{bmatrix}
	1.24 \\
	-0.88 \\
	0.62 \\
	-1.17
\end{bmatrix}
\]

\subsection*{2. Метод простых итераций:}

Для решения системы с использованием метода простых итераций применяется следующая формула обновления для каждого элемента:
\[
x_i^{(k+1)} = \sum_{j=1}^{n} P_{ij} \cdot x_j^{(k)} + g_i
\]
где \( x_i^{(k)} \) — это значение переменной на \(k\)-м шаге, а \( x_i^{(k+1)} \) — новое значение на \( (k+1) \)-м шаге.

Процесс продолжается до тех пор, пока разница между старыми и новыми значениями переменных не станет меньше заданной погрешности (в данном случае 0.001).


Функция работает следующим образом:

\begin{itemize}
	\item Инициализируется вектор \( x \) с нулевыми значениями.
	\item В цикле происходит итерационное обновление значений вектора \( x \) согласно формуле метода простых итераций.
	\item Итерации продолжаются до тех пор, пока разница между новыми и старыми значениями переменных не станет меньше заданной погрешности.
\end{itemize}

\subsection*{4. Решение системы:}

Применив метод простых итераций с матрицей коэффициентов \( P \) и вектором свободных членов \( g \), мы получаем решение системы уравнений.

\[
\text{Решение: } [2.070\pm 0.001, 0.151\pm 0.001, 1.090\pm 0.001, -0.663\pm 0.001]
\]
\textbf{Листинг программы: }
\lstinputlisting[language=Python, caption={Python код программы}]{simple_iter.py}


\section{LU-разложение}
\textbf{Задание 3}\\
Решить систему линейных уравнений 3-го порядка методом обратной матрицы с точностью до e=0.001
\begin{equation*}
	\begin{cases}
		4*x1+8*x2+7*x3=9       \\
		x1+2*x2+2*x3=2       \\
		2*x1+3*x2+x3=9   \\ 
		
	\end{cases}
\end{equation*}

Преобразуем матрицу $A$ в разложение $PA = LU$, где:

\[
P =
\begin{bmatrix}
	1 & 0 & 0 \\
	0 & 0 & 1 \\
	0 & 1 & 0
\end{bmatrix},
\quad
L =
\begin{bmatrix}
	1 & 0 & 0 \\
	0.5 & 1 & 0 \\
	0.25 & -1 & 1
\end{bmatrix},
\quad
U =
\begin{bmatrix}
	4 & 8 & 7 \\
	0 & -1 & -2.5 \\
	0 & 0 & 0.5
\end{bmatrix}.
\]



\subsection{Вычисление обратной матрицы}
Вычислим обратную матрицу по формуле 
\[
P L U \mathbf{x}_i = \mathbf{e}_i.
\]



Для решения системы сначала находим \( y_i \) с использованием прямого хода, затем решаем для \( x_i \) с использованием обратного хода.

 После того, как мы получим все столбцы \( \mathbf{x}_i \), они составляют столбцы обратной матрицы \( A^{-1} \).

\[ A^{-1} = 
\begin{bmatrix}
	-4 & 13 & 2 \\
	3 & -10 & -1 \\
	-1 & 4 & 0
\end{bmatrix}
\]
Таким образом, метод LU-разложения с перестановками позволяет эффективно вычислить обратную матрицу.

\subsection{Найдем корни системы}
По формуле вычисления корней через обратную матрциу
\[
	x = A^{-1}*b
\]
Где b - свободные члены системы

Подставив в формулу получаем корни системы
\[
\mathbf{x} =
\begin{bmatrix}
	8 \\
	-2 \\
	-1
\end{bmatrix}.
\]

\subsection{Листинг программы: }
\lstinputlisting[language=Python, caption={Python код программы}]{LDU.py}
\subsection{Проверка в Maxima}
	\begin{figure}[H] 
	\center
	\includegraphics [width=0.35\linewidth]{InvMatr.png}
	\caption{Правильное решение в Maxima}
\end{figure}


	\section{Задание 4}
	             \textbf{Задание 4} \\
	Решить систему нелийнейных уравнений 2-го порядка метод Ньютона с точность e=0.001: 
	Уравнение системы:
	
	\begin{equation*}
		\begin{cases}
			30*x^2+7*y^2-1=0            \\
			\sin(4*x-0,5*y)+5*x=0      \\
		\end{cases}	
	\end{equation*}
	
	\section*{Метод Ньютона}
	
	\begin{figure}[H] 
		\center
		\includegraphics [width=0.35\linewidth]{График функции.png}
		\caption{График функций}
	\end{figure}
	Возьмем за начальное приближение $x_0 = 0.1$ и $y_0 = 0.1$
	Метод Ньютона использует итерационный процесс:
	\[
	\begin{bmatrix} x_{k+1} \\ y_{k+1} \end{bmatrix} = 
	\begin{bmatrix} x_k \\ y_k \end{bmatrix} - J^{-1}(x_k, y_k) \cdot 
	\begin{bmatrix} f_1(x_k, y_k) \\ f_2(x_k, y_k) \end{bmatrix},
	\]
	где $J(x, y)$ — якобиан:
	\[
	J(x, y) =
	\begin{bmatrix} 
		\frac{\partial f_1}{\partial x} & \frac{\partial f_1}{\partial y} \\ 
		\frac{\partial f_2}{\partial x} & \frac{\partial f_2}{\partial y} 
	\end{bmatrix}.
	\]
	
	\section*{Частные производные}
	\[
	J(x, y) =
	\begin{bmatrix} 
		60x & 14y \\ 
		4\cos(4x - 0.5y) + 5 & -2\cos(4x - 0.5y)
	\end{bmatrix}.
	\]
	
	Определитель якобиана:
	\[
	\det J = (60x)(-2\cos(4x - 0.5y)) - (14y)(4\cos(4x - 0.5y) + 5).
	\]
	
	Обратная матрица:
	\[
	J^{-1} = \frac{1}{\det J} 
	\begin{bmatrix} 
		-2\cos(4x - 0.5y) & -14y \\ 
		- (4\cos(4x - 0.5y) + 5) & 60x
	\end{bmatrix}.
	\]
	
	\section*{Численное решение}
	Используя начальное приближение $(x_0, y_0) = (0.1, 0.1)$ и точность $\varepsilon = 10^{-3}$
	
	Решение:
	\[
	x \approx \text{0.020}\pm 0.001, \quad y \approx \text{0.375}\pm 0.001.
	\]
	\subsection{Листинг программы: }
	\lstinputlisting[language=Python, caption={Python код программы}]{Jacobian.py}
	\subsection{Проверка в Maxima}
	\begin{figure}[H] 
		\center
		\includegraphics [width=0.35\linewidth]{МаксимаНьютон.png}
		\caption{Правильное решение в Maxima}
	\end{figure}
	
	\section{Вывод}
	В ходе данной лабораторной работы, я смог ознакомлся с 3 методами решения линейных уравнений и 1 для решения нелинейных уравнений. А именно: метод Ньютона, метод Гаусса, LU разложение с перестановками и метод простых итераций. Эти методы были реализованы в программе с использованием высокого языка программирования (Python). Также вычесленные корни были проверены в Maxima, они сошлись с минимальной погрешностью.
\end{document}


